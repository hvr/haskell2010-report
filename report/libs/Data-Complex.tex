\chapter{\texttt{Data.Complex}}
\label{module:Data.Complex}
\haddockbeginheader
{\haddockverb\begin{verbatim}
module Data.Complex (
    Complex(:+),  realPart,  imagPart,  mkPolar,  cis,  polar,  magnitude, 
    phase,  conjugate
  ) where\end{verbatim}}
\haddockendheader

\section{Rectangular form
}
\begin{haddockdesc}
\item[\begin{tabular}{@{}l}
data\ RealFloat\ a\ =>\ Complex\ a
\end{tabular}]\haddockbegindoc
\haddockbeginconstrs
\haddockdecltt{=} & \haddockdecltt{!a :+ !a} & forms a complex number from its real and imaginary
 rectangular components.
 \\
\haddockendconstrs\par
Complex numbers are an algebraic type.
\par
For a complex number \haddocktt{z}, \haddocktt{abs\ z} is a number with the magnitude of \haddocktt{z},
 but oriented in the positive real direction, whereas \haddocktt{signum\ z}
 has the phase of \haddocktt{z}, but unit magnitude.
\par

\end{haddockdesc}
\begin{haddockdesc}
\item[\begin{tabular}{@{}l}
instance\ Typeable1\ Complex\\instance\ RealFloat\ a\ =>\ Eq\ (Complex\ a)\\instance\ RealFloat\ a\ =>\ Floating\ (Complex\ a)\\instance\ RealFloat\ a\ =>\ Fractional\ (Complex\ a)\\instance\ (Data\ a,\ RealFloat\ a)\ =>\ Data\ (Complex\ a)\\instance\ RealFloat\ a\ =>\ Num\ (Complex\ a)\\instance\ (Read\ a,\ RealFloat\ a)\ =>\ Read\ (Complex\ a)\\instance\ RealFloat\ a\ =>\ Show\ (Complex\ a)
\end{tabular}]
\end{haddockdesc}
\begin{haddockdesc}
\item[\begin{tabular}{@{}l}
realPart\ ::\ RealFloat\ a\ =>\ Complex\ a\ ->\ a
\end{tabular}]\haddockbegindoc
Extracts the real part of a complex number.
\par

\end{haddockdesc}
\begin{haddockdesc}
\item[\begin{tabular}{@{}l}
imagPart\ ::\ RealFloat\ a\ =>\ Complex\ a\ ->\ a
\end{tabular}]\haddockbegindoc
Extracts the imaginary part of a complex number.
\par

\end{haddockdesc}
\section{Polar form
}
\begin{haddockdesc}
\item[\begin{tabular}{@{}l}
mkPolar\ ::\ RealFloat\ a\ =>\ a\ ->\ a\ ->\ Complex\ a
\end{tabular}]\haddockbegindoc
Form a complex number from polar components of magnitude and phase.
\par

\end{haddockdesc}
\begin{haddockdesc}
\item[\begin{tabular}{@{}l}
cis\ ::\ RealFloat\ a\ =>\ a\ ->\ Complex\ a
\end{tabular}]\haddockbegindoc
\haddocktt{cis\ t} is a complex value with magnitude \haddocktt{1}
 and phase \haddocktt{t} (modulo \haddocktt{2*pi}).
\par

\end{haddockdesc}
\begin{haddockdesc}
\item[\begin{tabular}{@{}l}
polar\ ::\ RealFloat\ a\ =>\ Complex\ a\ ->\ (a,\ a)
\end{tabular}]\haddockbegindoc
The function \haddockid{polar} takes a complex number and
 returns a (magnitude, phase) pair in canonical form:
 the magnitude is nonnegative, and the phase in the range \haddocktt{(-pi,\ pi{\char 93}};
 if the magnitude is zero, then so is the phase.
\par

\end{haddockdesc}
\begin{haddockdesc}
\item[\begin{tabular}{@{}l}
magnitude\ ::\ RealFloat\ a\ =>\ Complex\ a\ ->\ a
\end{tabular}]\haddockbegindoc
The nonnegative magnitude of a complex number.
\par

\end{haddockdesc}
\begin{haddockdesc}
\item[\begin{tabular}{@{}l}
phase\ ::\ RealFloat\ a\ =>\ Complex\ a\ ->\ a
\end{tabular}]\haddockbegindoc
The phase of a complex number, in the range \haddocktt{(-pi,\ pi{\char 93}}.
 If the magnitude is zero, then so is the phase.
\par

\end{haddockdesc}
\section{Conjugate
}
\begin{haddockdesc}
\item[\begin{tabular}{@{}l}
conjugate\ ::\ RealFloat\ a\ =>\ Complex\ a\ ->\ Complex\ a
\end{tabular}]\haddockbegindoc
The conjugate of a complex number.
\par

\end{haddockdesc}
\section{Specification
}
\begin{quote}
{\haddockverb\begin{verbatim}
 module Complex(Complex((:+)), realPart, imagPart, conjugate, mkPolar,
                cis, polar, magnitude, phase)  where
 
 infix  6  :+
 
 data  (RealFloat a)     => Complex a = !a :+ !a  deriving (Eq,Read,Show)
 
 
 realPart, imagPart :: (RealFloat a) => Complex a -> a
 realPart (x:+y)	 =  x
 imagPart (x:+y)	 =  y
 
 conjugate	 :: (RealFloat a) => Complex a -> Complex a
 conjugate (x:+y) =  x :+ (-y)
 
 mkPolar		 :: (RealFloat a) => a -> a -> Complex a
 mkPolar r theta	 =  r * cos theta :+ r * sin theta
 
 cis		 :: (RealFloat a) => a -> Complex a
 cis theta	 =  cos theta :+ sin theta
 
 polar		 :: (RealFloat a) => Complex a -> (a,a)
 polar z		 =  (magnitude z, phase z)
 
 magnitude :: (RealFloat a) => Complex a -> a
 magnitude (x:+y) =  scaleFloat k
 		     (sqrt ((scaleFloat mk x)^2 + (scaleFloat mk y)^2))
 		    where k  = max (exponent x) (exponent y)
 		          mk = - k
 
 phase :: (RealFloat a) => Complex a -> a
 phase (0 :+ 0) = 0
 phase (x :+ y) = atan2 y x
 
 
 instance  (RealFloat a) => Num (Complex a)  where
     (x:+y) + (x':+y')	=  (x+x') :+ (y+y')
     (x:+y) - (x':+y')	=  (x-x') :+ (y-y')
     (x:+y) * (x':+y')	=  (x*x'-y*y') :+ (x*y'+y*x')
     negate (x:+y)	=  negate x :+ negate y
     abs z		=  magnitude z :+ 0
     signum 0		=  0
     signum z@(x:+y)	=  x/r :+ y/r  where r = magnitude z
     fromInteger n	=  fromInteger n :+ 0
 
 instance  (RealFloat a) => Fractional (Complex a)  where
     (x:+y) / (x':+y')	=  (x*x''+y*y'') / d :+ (y*x''-x*y'') / d
 			   where x'' = scaleFloat k x'
 				 y'' = scaleFloat k y'
 				 k   = - max (exponent x') (exponent y')
 				 d   = x'*x'' + y'*y''
 
     fromRational a	=  fromRational a :+ 0
 
 instance  (RealFloat a) => Floating (Complex a)	where
     pi             =  pi :+ 0
     exp (x:+y)     =  expx * cos y :+ expx * sin y
                       where expx = exp x
     log z          =  log (magnitude z) :+ phase z
 
     sqrt 0         =  0
     sqrt z@(x:+y)  =  u :+ (if y < 0 then -v else v)
                       where (u,v) = if x < 0 then (v',u') else (u',v')
                             v'    = abs y / (u'*2)
                             u'    = sqrt ((magnitude z + abs x) / 2)
 
     sin (x:+y)     =  sin x * cosh y :+ cos x * sinh y
     cos (x:+y)     =  cos x * cosh y :+ (- sin x * sinh y)
     tan (x:+y)     =  (sinx*coshy:+cosx*sinhy)/(cosx*coshy:+(-sinx*sinhy))
                       where sinx  = sin x
                             cosx  = cos x
                             sinhy = sinh y
                             coshy = cosh y
 
     sinh (x:+y)    =  cos y * sinh x :+ sin  y * cosh x
     cosh (x:+y)    =  cos y * cosh x :+ sin y * sinh x
     tanh (x:+y)    =  (cosy*sinhx:+siny*coshx)/(cosy*coshx:+siny*sinhx)
                       where siny  = sin y
                             cosy  = cos y
                             sinhx = sinh x
                             coshx = cosh x
 
     asin z@(x:+y)  =  y':+(-x')
                       where  (x':+y') = log (((-y):+x) + sqrt (1 - z*z))
     acos z@(x:+y)  =  y'':+(-x'')
                       where (x'':+y'') = log (z + ((-y'):+x'))
                             (x':+y')   = sqrt (1 - z*z)
     atan z@(x:+y)  =  y':+(-x')
                       where (x':+y') = log (((1-y):+x) / sqrt (1+z*z))
 
     asinh z        =  log (z + sqrt (1+z*z))
     acosh z        =  log (z + (z+1) * sqrt ((z-1)/(z+1)))
     atanh z        =  log ((1+z) / sqrt (1-z*z))
 
\end{verbatim}}
\end{quote}
