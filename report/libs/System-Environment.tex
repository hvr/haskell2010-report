\haddockmoduleheading{System.Environment}
\label{module:System.Environment}
\haddockbeginheader
{\haddockverb\begin{verbatim}
module System.Environment (
    getArgs,  getProgName,  getEnv
  ) where\end{verbatim}}
\haddockendheader

\begin{haddockdesc}
\item[\begin{tabular}{@{}l}
getArgs\ ::\ IO\ {\char 91}String{\char 93}
\end{tabular}]\haddockbegindoc
Computation \haddockid{getArgs} returns a list of the program's command
 line arguments (not including the program name).
\par

\end{haddockdesc}
\begin{haddockdesc}
\item[\begin{tabular}{@{}l}
getProgName\ ::\ IO\ String
\end{tabular}]\haddockbegindoc
Computation \haddockid{getProgName} returns the name of the program as it was
invoked.
\par
However, this is hard-to-impossible to implement on some non-Unix
OSes, so instead, for maximum portability, we just return the leafname
of the program as invoked. Even then there are some differences
between platforms: on Windows, for example, a program invoked as foo
is probably really \haddocktt{FOO.EXE}, and that is what \haddockid{getProgName} will return.
\par

\end{haddockdesc}
\begin{haddockdesc}
\item[\begin{tabular}{@{}l}
getEnv\ ::\ String\ ->\ IO\ String
\end{tabular}]\haddockbegindoc
Computation \haddockid{getEnv} \haddocktt{var} returns the value
 of the environment variable \haddocktt{var}.  
\par
This computation may fail with:
\par
\begin{itemize}
\item
 \haddocktt{System.IO.Error.isDoesNotExistError} if the environment variable
    does not exist.
\par

\end{itemize}

\end{haddockdesc}