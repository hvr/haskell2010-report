\chapter{\texttt{Foreign.Marshal.Utils}}
\label{module:Foreign.Marshal.Utils}
\haddockbeginheader
{\haddockverb\begin{verbatim}
module Foreign.Marshal.Utils (
    with,  new,  fromBool,  toBool,  maybeNew,  maybeWith,  maybePeek, 
    withMany,  copyBytes,  moveBytes
  ) where\end{verbatim}}
\haddockendheader

\section{General marshalling utilities
}
\subsection{Combined allocation and marshalling
}
\begin{haddockdesc}
\item[\begin{tabular}{@{}l}
with\ ::\ Storable\ a\ =>\ a\ ->\ (Ptr\ a\ ->\ IO\ b)\ ->\ IO\ b
\end{tabular}]\haddockbegindoc
\haddocktt{with\ val\ f} executes the computation \haddocktt{f}, passing as argument
 a pointer to a temporarily allocated block of memory into which
 \haddocktt{val} has been marshalled (the combination of \haddockid{alloca} and \haddockid{poke}).
\par
The memory is freed when \haddocktt{f} terminates (either normally or via an
 exception), so the pointer passed to \haddocktt{f} must \emph{not} be used after this.
\par

\end{haddockdesc}
\begin{haddockdesc}
\item[\begin{tabular}{@{}l}
new\ ::\ Storable\ a\ =>\ a\ ->\ IO\ (Ptr\ a)
\end{tabular}]\haddockbegindoc
Allocate a block of memory and marshal a value into it
 (the combination of \haddockid{malloc} and \haddockid{poke}).
 The size of the area allocated is determined by the \haddocktt{Foreign.Storable.sizeOf}
 method from the instance of \haddockid{Storable} for the appropriate type.
\par
The memory may be deallocated using \haddocktt{Foreign.Marshal.Alloc.free} or
 \haddocktt{Foreign.Marshal.Alloc.finalizerFree} when no longer required.
\par

\end{haddockdesc}
\subsection{Marshalling of Boolean values (non-zero corresponds to \haddockid{True})
}
\begin{haddockdesc}
\item[\begin{tabular}{@{}l}
fromBool\ ::\ Num\ a\ =>\ Bool\ ->\ a
\end{tabular}]\haddockbegindoc
Convert a Haskell \haddockid{Bool} to its numeric representation
\par

\end{haddockdesc}
\begin{haddockdesc}
\item[\begin{tabular}{@{}l}
toBool\ ::\ Num\ a\ =>\ a\ ->\ Bool
\end{tabular}]\haddockbegindoc
Convert a Boolean in numeric representation to a Haskell value
\par

\end{haddockdesc}
\subsection{Marshalling of Maybe values
}
\begin{haddockdesc}
\item[\begin{tabular}{@{}l}
maybeNew\ ::\ (a\ ->\ IO\ (Ptr\ a))\ ->\ Maybe\ a\ ->\ IO\ (Ptr\ a)
\end{tabular}]\haddockbegindoc
Allocate storage and marshall a storable value wrapped into a \haddockid{Maybe}
\par
\begin{itemize}
\item
 the \haddockid{nullPtr} is used to represent \haddockid{Nothing}
\par

\end{itemize}

\end{haddockdesc}
\begin{haddockdesc}
\item[\begin{tabular}{@{}l}
maybeWith\ ::\ (a\ ->\ (Ptr\ b\ ->\ IO\ c)\ ->\ IO\ c)\\\ \ \ \ \ \ \ \ \ \ \ \ \ ->\ Maybe\ a\ ->\ (Ptr\ b\ ->\ IO\ c)\ ->\ IO\ c
\end{tabular}]\haddockbegindoc
Converts a \haddocktt{withXXX} combinator into one marshalling a value wrapped
 into a \haddockid{Maybe}, using \haddockid{nullPtr} to represent \haddockid{Nothing}.
\par

\end{haddockdesc}
\begin{haddockdesc}
\item[\begin{tabular}{@{}l}
maybePeek\ ::\ (Ptr\ a\ ->\ IO\ b)\ ->\ Ptr\ a\ ->\ IO\ (Maybe\ b)
\end{tabular}]\haddockbegindoc
Convert a peek combinator into a one returning \haddockid{Nothing} if applied to a
 \haddockid{nullPtr} 
\par

\end{haddockdesc}
\subsection{Marshalling lists of storable objects
}
\begin{haddockdesc}
\item[\begin{tabular}{@{}l}
withMany\ ::\ (a\ ->\ (b\ ->\ res)\ ->\ res)\ ->\ {\char 91}a{\char 93}\ ->\ ({\char 91}b{\char 93}\ ->\ res)\ ->\ res
\end{tabular}]\haddockbegindoc
Replicates a \haddocktt{withXXX} combinator over a list of objects, yielding a list of
 marshalled objects
\par

\end{haddockdesc}
\subsection{Haskellish interface to memcpy and memmove
}
(argument order: destination, source)
\par

\begin{haddockdesc}
\item[\begin{tabular}{@{}l}
copyBytes\ ::\ Ptr\ a\ ->\ Ptr\ a\ ->\ Int\ ->\ IO\ ()
\end{tabular}]\haddockbegindoc
Copies the given number of bytes from the second area (source) into the
 first (destination); the copied areas may \emph{not} overlap
\par

\end{haddockdesc}
\begin{haddockdesc}
\item[\begin{tabular}{@{}l}
moveBytes\ ::\ Ptr\ a\ ->\ Ptr\ a\ ->\ Int\ ->\ IO\ ()
\end{tabular}]\haddockbegindoc
Copies the given number of bytes from the second area (source) into the
 first (destination); the copied areas \emph{may} overlap
\par

\end{haddockdesc}