\haddockmoduleheading{Foreign.Marshal.Alloc}
\label{module:Foreign.Marshal.Alloc}
\haddockbeginheader
{\haddockverb\begin{verbatim}
module Foreign.Marshal.Alloc (
    alloca,  allocaBytes,  malloc,  mallocBytes,  realloc,  reallocBytes, 
    free,  finalizerFree
  ) where\end{verbatim}}
\haddockendheader

The module \haddocktt{Foreign.Marshal.Alloc} provides operations to allocate and
deallocate blocks of raw memory (i.e., unstructured chunks of memory
outside of the area maintained by the Haskell storage manager).  These
memory blocks are commonly used to pass compound data structures to
foreign functions or to provide space in which compound result values
are obtained from foreign functions.
\par
If any of the allocation functions fails, a value of \haddocktt{nullPtr} is
produced.  If \haddockid{free} or \haddockid{reallocBytes} is applied to a memory area
that has been allocated with \haddockid{alloca} or \haddockid{allocaBytes}, the
behaviour is undefined.  Any further access to memory areas allocated with
\haddockid{alloca} or \haddockid{allocaBytes}, after the computation that was passed to
the allocation function has terminated, leads to undefined behaviour.  Any
further access to the memory area referenced by a pointer passed to
\haddockid{realloc}, \haddockid{reallocBytes}, or \haddockid{free} entails undefined
behaviour.
\par

\section{Memory allocation
}
\subsection{Local allocation
}
\begin{haddockdesc}
\item[\begin{tabular}{@{}l}
alloca\ ::\ Storable\ a\ =>\ (Ptr\ a\ ->\ IO\ b)\ ->\ IO\ b
\end{tabular}]\haddockbegindoc
\haddocktt{alloca\ f} executes the computation \haddocktt{f}, passing as argument
 a pointer to a temporarily allocated block of memory sufficient to
 hold values of type \haddocktt{a}.
\par
The memory is freed when \haddocktt{f} terminates (either normally or via an
 exception), so the pointer passed to \haddocktt{f} must \emph{not} be used after this.
\par

\end{haddockdesc}
\begin{haddockdesc}
\item[\begin{tabular}{@{}l}
allocaBytes\ ::\ Int\ ->\ (Ptr\ a\ ->\ IO\ b)\ ->\ IO\ b
\end{tabular}]\haddockbegindoc
\haddocktt{allocaBytes\ n\ f} executes the computation \haddocktt{f}, passing as argument
 a pointer to a temporarily allocated block of memory of \haddocktt{n} bytes.
 The block of memory is sufficiently aligned for any of the basic
 foreign types that fits into a memory block of the allocated size.
\par
The memory is freed when \haddocktt{f} terminates (either normally or via an
 exception), so the pointer passed to \haddocktt{f} must \emph{not} be used after this.
\par

\end{haddockdesc}
\subsection{Dynamic allocation
}
\begin{haddockdesc}
\item[\begin{tabular}{@{}l}
malloc\ ::\ Storable\ a\ =>\ IO\ (Ptr\ a)
\end{tabular}]\haddockbegindoc
Allocate a block of memory that is sufficient to hold values of type
 \haddocktt{a}.  The size of the area allocated is determined by the \haddockid{sizeOf}
 method from the instance of \haddockid{Storable} for the appropriate type.
\par
The memory may be deallocated using \haddockid{free} or \haddockid{finalizerFree} when
 no longer required.
\par

\end{haddockdesc}
\begin{haddockdesc}
\item[\begin{tabular}{@{}l}
mallocBytes\ ::\ Int\ ->\ IO\ (Ptr\ a)
\end{tabular}]\haddockbegindoc
Allocate a block of memory of the given number of bytes.
 The block of memory is sufficiently aligned for any of the basic
 foreign types that fits into a memory block of the allocated size.
\par
The memory may be deallocated using \haddockid{free} or \haddockid{finalizerFree} when
 no longer required.
\par

\end{haddockdesc}
\begin{haddockdesc}
\item[\begin{tabular}{@{}l}
realloc\ ::\ Storable\ b\ =>\ Ptr\ a\ ->\ IO\ (Ptr\ b)
\end{tabular}]\haddockbegindoc
Resize a memory area that was allocated with \haddockid{malloc} or \haddockid{mallocBytes}
 to the size needed to store values of type \haddocktt{b}.  The returned pointer
 may refer to an entirely different memory area, but will be suitably
 aligned to hold values of type \haddocktt{b}.  The contents of the referenced
 memory area will be the same as of the original pointer up to the
 minimum of the original size and the size of values of type \haddocktt{b}.
\par
If the argument to \haddockid{realloc} is \haddockid{nullPtr}, \haddockid{realloc} behaves like
 \haddockid{malloc}.
\par

\end{haddockdesc}
\begin{haddockdesc}
\item[\begin{tabular}{@{}l}
reallocBytes\ ::\ Ptr\ a\ ->\ Int\ ->\ IO\ (Ptr\ a)
\end{tabular}]\haddockbegindoc
Resize a memory area that was allocated with \haddockid{malloc} or \haddockid{mallocBytes}
 to the given size.  The returned pointer may refer to an entirely
 different memory area, but will be sufficiently aligned for any of the
 basic foreign types that fits into a memory block of the given size.
 The contents of the referenced memory area will be the same as of
 the original pointer up to the minimum of the original size and the
 given size.
\par
If the pointer argument to \haddockid{reallocBytes} is \haddockid{nullPtr}, \haddockid{reallocBytes}
 behaves like \haddockid{malloc}.  If the requested size is 0, \haddockid{reallocBytes}
 behaves like \haddockid{free}.
\par

\end{haddockdesc}
\begin{haddockdesc}
\item[\begin{tabular}{@{}l}
free\ ::\ Ptr\ a\ ->\ IO\ ()
\end{tabular}]\haddockbegindoc
Free a block of memory that was allocated with \haddockid{malloc},
 \haddockid{mallocBytes}, \haddockid{realloc}, \haddockid{reallocBytes}, \haddocktt{Foreign.Marshal.Utils.new}
 or any of the \haddocktt{new}\emph{X} functions in \haddocktt{Foreign.Marshal.Array} or
 \haddocktt{Foreign.C.String}.
\par

\end{haddockdesc}
\begin{haddockdesc}
\item[\begin{tabular}{@{}l}
finalizerFree\ ::\ FinalizerPtr\ a
\end{tabular}]\haddockbegindoc
A pointer to a foreign function equivalent to \haddockid{free}, which may be
 used as a finalizer (cf \haddocktt{Foreign.ForeignPtr.ForeignPtr}) for storage
 allocated with \haddockid{malloc}, \haddockid{mallocBytes}, \haddockid{realloc} or \haddockid{reallocBytes}.
\par

\end{haddockdesc}