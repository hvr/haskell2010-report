\chapter{\texttt{System.IO}}
\label{module:System.IO}
\haddockbeginheader
{\haddockverb\begin{verbatim}
module System.IO (
    IO,  fixIO,  FilePath,  Handle,  stdin,  stdout,  stderr,  withFile, 
    openFile,  IOMode(ReadMode, WriteMode, AppendMode, ReadWriteMode),  hClose, 
    readFile,  writeFile,  appendFile,  hFileSize,  hSetFileSize,  hIsEOF, 
    isEOF,  BufferMode(NoBuffering, LineBuffering, BlockBuffering), 
    hSetBuffering,  hGetBuffering,  hFlush,  hGetPosn,  hSetPosn,  HandlePosn, 
    hSeek,  SeekMode(AbsoluteSeek, RelativeSeek, SeekFromEnd),  hTell, 
    hIsOpen,  hIsClosed,  hIsReadable,  hIsWritable,  hIsSeekable, 
    hIsTerminalDevice,  hSetEcho,  hGetEcho,  hShow,  hWaitForInput,  hReady, 
    hGetChar,  hGetLine,  hLookAhead,  hGetContents,  hPutChar,  hPutStr, 
    hPutStrLn,  hPrint,  interact,  putChar,  putStr,  putStrLn,  print, 
    getChar,  getLine,  getContents,  readIO,  readLn
  ) where\end{verbatim}}
\haddockendheader

\section{The IO monad
}
\begin{haddockdesc}
\item[\begin{tabular}{@{}l}
data\ IO\ a
\end{tabular}]\haddockbegindoc
A value of type \haddocktt{IO\ a} is a computation which, when performed,
does some I/O before returning a value of type \haddocktt{a}.
\par
There is really only one way to "perform" an I/O action: bind it to
\haddocktt{Main.main} in your program.  When your program is run, the I/O will
be performed.  It isn't possible to perform I/O from an arbitrary
function, unless that function is itself in the \haddockid{IO} monad and called
at some point, directly or indirectly, from \haddocktt{Main.main}.
\par
\haddockid{IO} is a monad, so \haddockid{IO} actions can be combined using either the do-notation
or the \haddocktt{>>} and \haddocktt{>>=} operations from the \haddocktt{Monad} class.
\par

\end{haddockdesc}
\begin{haddockdesc}
\item[\begin{tabular}{@{}l}
instance\ Monad\ IO\\instance\ Functor\ IO
\end{tabular}]
\end{haddockdesc}
\begin{haddockdesc}
\item[\begin{tabular}{@{}l}
fixIO\ ::\ (a\ ->\ IO\ a)\ ->\ IO\ a
\end{tabular}]
\end{haddockdesc}
\section{Files and handles
}
\begin{haddockdesc}
\item[\begin{tabular}{@{}l}
type\ FilePath\ =\ String
\end{tabular}]\haddockbegindoc
File and directory names are values of type \haddockid{String}, whose precise
 meaning is operating system dependent. Files can be opened, yielding a
 handle which can then be used to operate on the contents of that file.
\par

\end{haddockdesc}
\begin{haddockdesc}
\item[\begin{tabular}{@{}l}
data\ Handle
\end{tabular}]\haddockbegindoc
Haskell defines operations to read and write characters from and to files,
 represented by values of type \haddocktt{Handle}.  Each value of this type is a
 \emph{handle}: a record used by the Haskell run-time system to \emph{manage} I/O
 with file system objects.  A handle has at least the following properties:
\par
\begin{itemize}
\item
 whether it manages input or output or both;
\par

\item
 whether it is \emph{open}, \emph{closed} or \emph{semi-closed};
\par

\item
 whether the object is seekable;
\par

\item
 whether buffering is disabled, or enabled on a line or block basis;
\par

\item
 a buffer (whose length may be zero).
\par

\end{itemize}
Most handles will also have a current I/O position indicating where the next
 input or output operation will occur.  A handle is \emph{readable} if it
 manages only input or both input and output; likewise, it is \emph{writable} if
 it manages only output or both input and output.  A handle is \emph{open} when
 first allocated.
 Once it is closed it can no longer be used for either input or output,
 though an implementation cannot re-use its storage while references
 remain to it.  Handles are in the \haddockid{Show} and \haddockid{Eq} classes.  The string
 produced by showing a handle is system dependent; it should include
 enough information to identify the handle for debugging.  A handle is
 equal according to \haddockid{==} only to itself; no attempt
 is made to compare the internal state of different handles for equality.
\par

\end{haddockdesc}
\begin{haddockdesc}
\item[\begin{tabular}{@{}l}
instance\ Eq\ Handle\\instance\ Show\ Handle\\instance\ Typeable\ Handle
\end{tabular}]
\end{haddockdesc}
\subsection{Standard handles
}
Three handles are allocated during program initialisation,
 and are initially open.
\par

\begin{haddockdesc}
\item[\begin{tabular}{@{}l}
stdin\ ::\ Handle
\end{tabular}]\haddockbegindoc
A handle managing input from the Haskell program's standard input channel.
\par

\end{haddockdesc}
\begin{haddockdesc}
\item[\begin{tabular}{@{}l}
stdout\ ::\ Handle
\end{tabular}]\haddockbegindoc
A handle managing output to the Haskell program's standard output channel.
\par

\end{haddockdesc}
\begin{haddockdesc}
\item[\begin{tabular}{@{}l}
stderr\ ::\ Handle
\end{tabular}]\haddockbegindoc
A handle managing output to the Haskell program's standard error channel.
\par

\end{haddockdesc}
\section{Opening and closing files
}
\subsection{Opening files
}
\begin{haddockdesc}
\item[\begin{tabular}{@{}l}
withFile\ ::\ FilePath\ ->\ IOMode\ ->\ (Handle\ ->\ IO\ r)\ ->\ IO\ r
\end{tabular}]\haddockbegindoc
\haddocktt{withFile\ name\ mode\ act} opens a file using \haddockid{openFile} and passes
 the resulting handle to the computation \haddocktt{act}.  The handle will be
 closed on exit from \haddockid{withFile}, whether by normal termination or by
 raising an exception.
\par

\end{haddockdesc}
\begin{haddockdesc}
\item[\begin{tabular}{@{}l}
openFile\ ::\ FilePath\ ->\ IOMode\ ->\ IO\ Handle
\end{tabular}]\haddockbegindoc
Computation \haddockid{openFile} \haddocktt{file\ mode} allocates and returns a new, open
 handle to manage the file \haddocktt{file}.  It manages input if \haddocktt{mode}
 is \haddockid{ReadMode}, output if \haddocktt{mode} is \haddockid{WriteMode} or \haddockid{AppendMode},
 and both input and output if mode is \haddockid{ReadWriteMode}.
\par
If the file does not exist and it is opened for output, it should be
 created as a new file.  If \haddocktt{mode} is \haddockid{WriteMode} and the file
 already exists, then it should be truncated to zero length.
 Some operating systems delete empty files, so there is no guarantee
 that the file will exist following an \haddockid{openFile} with \haddocktt{mode}
 \haddockid{WriteMode} unless it is subsequently written to successfully.
 The handle is positioned at the end of the file if \haddocktt{mode} is
 \haddockid{AppendMode}, and otherwise at the beginning (in which case its
 internal position is 0).
 The initial buffer mode is implementation-dependent.
\par
This operation may fail with:
\par
\begin{itemize}
\item
 \haddocktt{isAlreadyInUseError} if the file is already open and cannot be reopened;
\par

\item
 \haddocktt{isDoesNotExistError} if the file does not exist; or
\par

\item
 \haddocktt{isPermissionError} if the user does not have permission to open the file.
\par

\end{itemize}

\end{haddockdesc}
\begin{haddockdesc}
\item[\begin{tabular}{@{}l}
data\ IOMode
\end{tabular}]\haddockbegindoc
\haddockbeginconstrs
\haddockdecltt{=} & \haddockdecltt{ReadMode} & \\
\haddockdecltt{|} & \haddockdecltt{WriteMode} & \\
\haddockdecltt{|} & \haddockdecltt{AppendMode} & \\
\haddockdecltt{|} & \haddockdecltt{ReadWriteMode} & \\
\haddockendconstrs\par
See \haddocktt{System.IO.openFile}
\par

\end{haddockdesc}
\begin{haddockdesc}
\item[\begin{tabular}{@{}l}
instance\ Enum\ IOMode\\instance\ Eq\ IOMode\\instance\ Ord\ IOMode\\instance\ Read\ IOMode\\instance\ Show\ IOMode\\instance\ Ix\ IOMode
\end{tabular}]
\end{haddockdesc}
\subsection{Closing files
}
\begin{haddockdesc}
\item[\begin{tabular}{@{}l}
hClose\ ::\ Handle\ ->\ IO\ ()
\end{tabular}]\haddockbegindoc
Computation \haddockid{hClose} \haddocktt{hdl} makes handle \haddocktt{hdl} closed.  Before the
 computation finishes, if \haddocktt{hdl} is writable its buffer is flushed as
 for \haddockid{hFlush}.
 Performing \haddockid{hClose} on a handle that has already been closed has no effect; 
 doing so is not an error.  All other operations on a closed handle will fail.
 If \haddockid{hClose} fails for any reason, any further operations (apart from
 \haddockid{hClose}) on the handle will still fail as if \haddocktt{hdl} had been successfully
 closed.
\par

\end{haddockdesc}
\subsection{Special cases
}
These functions are also exported by the \haddocktt{Prelude}.
\par

\begin{haddockdesc}
\item[\begin{tabular}{@{}l}
readFile\ ::\ FilePath\ ->\ IO\ String
\end{tabular}]\haddockbegindoc
The \haddockid{readFile} function reads a file and
 returns the contents of the file as a string.
 The file is read lazily, on demand, as with \haddockid{getContents}.
\par

\end{haddockdesc}
\begin{haddockdesc}
\item[\begin{tabular}{@{}l}
writeFile\ ::\ FilePath\ ->\ String\ ->\ IO\ ()
\end{tabular}]\haddockbegindoc
The computation \haddockid{writeFile} \haddocktt{file\ str} function writes the string \haddocktt{str},
 to the file \haddocktt{file}.
\par

\end{haddockdesc}
\begin{haddockdesc}
\item[\begin{tabular}{@{}l}
appendFile\ ::\ FilePath\ ->\ String\ ->\ IO\ ()
\end{tabular}]\haddockbegindoc
The computation \haddockid{appendFile} \haddocktt{file\ str} function appends the string \haddocktt{str},
 to the file \haddocktt{file}.
\par
Note that \haddockid{writeFile} and \haddockid{appendFile} write a literal string
 to a file.  To write a value of any printable type, as with \haddockid{print},
 use the \haddockid{show} function to convert the value to a string first.
\par
\begin{quote}
{\haddockverb\begin{verbatim}
 main = appendFile "squares" (show [(x,x*x) | x <- [0,0.1..2]])
\end{verbatim}}
\end{quote}

\end{haddockdesc}
\subsection{File locking
}
Implementations should enforce as far as possible, at least locally to the
 Haskell process, multiple-reader single-writer locking on files.
 That is, \emph{there may either be many handles on the same file which manage input, or just one handle on the file which manages output}.  If any
 open or semi-closed handle is managing a file for output, no new
 handle can be allocated for that file.  If any open or semi-closed
 handle is managing a file for input, new handles can only be allocated
 if they do not manage output.  Whether two files are the same is
 implementation-dependent, but they should normally be the same if they
 have the same absolute path name and neither has been renamed, for
 example.
\par
\emph{Warning}: the \haddockid{readFile} operation holds a semi-closed handle on
 the file until the entire contents of the file have been consumed.
 It follows that an attempt to write to a file (using \haddockid{writeFile}, for
 example) that was earlier opened by \haddockid{readFile} will usually result in
 failure with \haddocktt{System.IO.Error.isAlreadyInUseError}.
\par

\section{Operations on handles
}
\subsection{Determining and changing the size of a file
}
\begin{haddockdesc}
\item[\begin{tabular}{@{}l}
hFileSize\ ::\ Handle\ ->\ IO\ Integer
\end{tabular}]\haddockbegindoc
For a handle \haddocktt{hdl} which attached to a physical file,
 \haddockid{hFileSize} \haddocktt{hdl} returns the size of that file in 8-bit bytes.
\par

\end{haddockdesc}
\begin{haddockdesc}
\item[\begin{tabular}{@{}l}
hSetFileSize\ ::\ Handle\ ->\ Integer\ ->\ IO\ ()
\end{tabular}]\haddockbegindoc
\haddockid{hSetFileSize} \haddocktt{hdl} \haddocktt{size} truncates the physical file with handle \haddocktt{hdl} to \haddocktt{size} bytes.
\par

\end{haddockdesc}
\subsection{Detecting the end of input
}
\begin{haddockdesc}
\item[\begin{tabular}{@{}l}
hIsEOF\ ::\ Handle\ ->\ IO\ Bool
\end{tabular}]\haddockbegindoc
For a readable handle \haddocktt{hdl}, \haddockid{hIsEOF} \haddocktt{hdl} returns
 \haddockid{True} if no further input can be taken from \haddocktt{hdl} or for a
 physical file, if the current I/O position is equal to the length of
 the file.  Otherwise, it returns \haddockid{False}.
\par
NOTE: \haddockid{hIsEOF} may block, because it has to attempt to read from
 the stream to determine whether there is any more data to be read.
\par

\end{haddockdesc}
\begin{haddockdesc}
\item[\begin{tabular}{@{}l}
isEOF\ ::\ IO\ Bool
\end{tabular}]\haddockbegindoc
The computation \haddockid{isEOF} is identical to \haddockid{hIsEOF},
 except that it works only on \haddockid{stdin}.
\par

\end{haddockdesc}
\subsection{Buffering operations
}
\begin{haddockdesc}
\item[\begin{tabular}{@{}l}
data\ BufferMode
\end{tabular}]\haddockbegindoc
\haddockbeginconstrs
\haddockdecltt{=} & \haddockdecltt{NoBuffering} & buffering is disabled if possible.
 \\
\haddockdecltt{|} & \haddockdecltt{LineBuffering} & line-buffering should be enabled if possible.
 \\
\haddockdecltt{|} & \haddockdecltt{BlockBuffering (Maybe Int)} & block-buffering should be enabled if possible.
 The size of the buffer is \haddocktt{n} items if the argument
 is \haddockid{Just} \haddocktt{n} and is otherwise implementation-dependent.
 \\
\haddockendconstrs\par
Three kinds of buffering are supported: line-buffering, 
 block-buffering or no-buffering.  These modes have the following
 effects. For output, items are written out, or \emph{flushed},
 from the internal buffer according to the buffer mode:
\par
\begin{itemize}
\item
 \emph{line-buffering}: the entire output buffer is flushed
    whenever a newline is output, the buffer overflows, 
    a \haddocktt{System.IO.hFlush} is issued, or the handle is closed.
\par

\item
 \emph{block-buffering}: the entire buffer is written out whenever it
    overflows, a \haddocktt{System.IO.hFlush} is issued, or the handle is closed.
\par

\item
 \emph{no-buffering}: output is written immediately, and never stored
    in the buffer.
\par

\end{itemize}
An implementation is free to flush the buffer more frequently,
 but not less frequently, than specified above.
 The output buffer is emptied as soon as it has been written out.
\par
Similarly, input occurs according to the buffer mode for the handle:
\par
\begin{itemize}
\item
 \emph{line-buffering}: when the buffer for the handle is not empty,
    the next item is obtained from the buffer; otherwise, when the
    buffer is empty, characters up to and including the next newline
    character are read into the buffer.  No characters are available
    until the newline character is available or the buffer is full.
\par

\item
 \emph{block-buffering}: when the buffer for the handle becomes empty,
    the next block of data is read into the buffer.
\par

\item
 \emph{no-buffering}: the next input item is read and returned.
    The \haddocktt{System.IO.hLookAhead} operation implies that even a no-buffered
    handle may require a one-character buffer.
\par

\end{itemize}
The default buffering mode when a handle is opened is
 implementation-dependent and may depend on the file system object
 which is attached to that handle.
 For most implementations, physical files will normally be block-buffered 
 and terminals will normally be line-buffered.
\par

\end{haddockdesc}
\begin{haddockdesc}
\item[\begin{tabular}{@{}l}
instance\ Eq\ BufferMode\\instance\ Ord\ BufferMode\\instance\ Read\ BufferMode\\instance\ Show\ BufferMode
\end{tabular}]
\end{haddockdesc}
\begin{haddockdesc}
\item[\begin{tabular}{@{}l}
hSetBuffering\ ::\ Handle\ ->\ BufferMode\ ->\ IO\ ()
\end{tabular}]\haddockbegindoc
Computation \haddockid{hSetBuffering} \haddocktt{hdl\ mode} sets the mode of buffering for
 handle \haddocktt{hdl} on subsequent reads and writes.
\par
If the buffer mode is changed from \haddockid{BlockBuffering} or
 \haddockid{LineBuffering} to \haddockid{NoBuffering}, then
\par
\begin{itemize}
\item
 if \haddocktt{hdl} is writable, the buffer is flushed as for \haddockid{hFlush};
\par

\item
 if \haddocktt{hdl} is not writable, the contents of the buffer is discarded.
\par

\end{itemize}
This operation may fail with:
\par
\begin{itemize}
\item
 \haddocktt{isPermissionError} if the handle has already been used for reading
    or writing and the implementation does not allow the buffering mode
    to be changed.
\par

\end{itemize}

\end{haddockdesc}
\begin{haddockdesc}
\item[\begin{tabular}{@{}l}
hGetBuffering\ ::\ Handle\ ->\ IO\ BufferMode
\end{tabular}]\haddockbegindoc
Computation \haddockid{hGetBuffering} \haddocktt{hdl} returns the current buffering mode
 for \haddocktt{hdl}.
\par

\end{haddockdesc}
\begin{haddockdesc}
\item[\begin{tabular}{@{}l}
hFlush\ ::\ Handle\ ->\ IO\ ()
\end{tabular}]\haddockbegindoc
The action \haddockid{hFlush} \haddocktt{hdl} causes any items buffered for output
 in handle \haddocktt{hdl} to be sent immediately to the operating system.
\par
This operation may fail with:
\par
\begin{itemize}
\item
 \haddocktt{isFullError} if the device is full;
\par

\item
 \haddocktt{isPermissionError} if a system resource limit would be exceeded.
    It is unspecified whether the characters in the buffer are discarded
    or retained under these circumstances.
\par

\end{itemize}

\end{haddockdesc}
\subsection{Repositioning handles
}
\begin{haddockdesc}
\item[\begin{tabular}{@{}l}
hGetPosn\ ::\ Handle\ ->\ IO\ HandlePosn
\end{tabular}]\haddockbegindoc
Computation \haddockid{hGetPosn} \haddocktt{hdl} returns the current I/O position of
 \haddocktt{hdl} as a value of the abstract type \haddockid{HandlePosn}.
\par

\end{haddockdesc}
\begin{haddockdesc}
\item[\begin{tabular}{@{}l}
hSetPosn\ ::\ HandlePosn\ ->\ IO\ ()
\end{tabular}]\haddockbegindoc
If a call to \haddockid{hGetPosn} \haddocktt{hdl} returns a position \haddocktt{p},
 then computation \haddockid{hSetPosn} \haddocktt{p} sets the position of \haddocktt{hdl}
 to the position it held at the time of the call to \haddockid{hGetPosn}.
\par
This operation may fail with:
\par
\begin{itemize}
\item
 \haddocktt{isPermissionError} if a system resource limit would be exceeded.
\par

\end{itemize}

\end{haddockdesc}
\begin{haddockdesc}
\item[\begin{tabular}{@{}l}
data\ HandlePosn
\end{tabular}]
\end{haddockdesc}
\begin{haddockdesc}
\item[\begin{tabular}{@{}l}
instance\ Eq\ HandlePosn\\instance\ Show\ HandlePosn
\end{tabular}]
\end{haddockdesc}
\begin{haddockdesc}
\item[\begin{tabular}{@{}l}
hSeek\ ::\ Handle\ ->\ SeekMode\ ->\ Integer\ ->\ IO\ ()
\end{tabular}]\haddockbegindoc
Computation \haddockid{hSeek} \haddocktt{hdl\ mode\ i} sets the position of handle
 \haddocktt{hdl} depending on \haddocktt{mode}.
 The offset \haddocktt{i} is given in terms of 8-bit bytes.
\par
If \haddocktt{hdl} is block- or line-buffered, then seeking to a position which is not
 in the current buffer will first cause any items in the output buffer to be
 written to the device, and then cause the input buffer to be discarded.
 Some handles may not be seekable (see \haddockid{hIsSeekable}), or only support a
 subset of the possible positioning operations (for instance, it may only
 be possible to seek to the end of a tape, or to a positive offset from
 the beginning or current position).
 It is not possible to set a negative I/O position, or for
 a physical file, an I/O position beyond the current end-of-file.
\par
This operation may fail with:
\par
\begin{itemize}
\item
 \haddocktt{isIllegalOperationError} if the Handle is not seekable, or does
    not support the requested seek mode.
  * \haddocktt{isPermissionError} if a system resource limit would be exceeded.
\par

\end{itemize}

\end{haddockdesc}
\begin{haddockdesc}
\item[\begin{tabular}{@{}l}
data\ SeekMode
\end{tabular}]\haddockbegindoc
\haddockbeginconstrs
\haddockdecltt{=} & \haddockdecltt{AbsoluteSeek} & the position of \haddocktt{hdl} is set to \haddocktt{i}.
 \\
\haddockdecltt{|} & \haddockdecltt{RelativeSeek} & the position of \haddocktt{hdl} is set to offset \haddocktt{i}
 from the current position.
 \\
\haddockdecltt{|} & \haddockdecltt{SeekFromEnd} & the position of \haddocktt{hdl} is set to offset \haddocktt{i}
 from the end of the file.
 \\
\haddockendconstrs\par
A mode that determines the effect of \haddocktt{hSeek} \haddocktt{hdl\ mode\ i}.
\par

\end{haddockdesc}
\begin{haddockdesc}
\item[\begin{tabular}{@{}l}
instance\ Enum\ SeekMode\\instance\ Eq\ SeekMode\\instance\ Ord\ SeekMode\\instance\ Read\ SeekMode\\instance\ Show\ SeekMode\\instance\ Ix\ SeekMode
\end{tabular}]
\end{haddockdesc}
\begin{haddockdesc}
\item[\begin{tabular}{@{}l}
hTell\ ::\ Handle\ ->\ IO\ Integer
\end{tabular}]\haddockbegindoc
Computation \haddockid{hTell} \haddocktt{hdl} returns the current position of the
 handle \haddocktt{hdl}, as the number of bytes from the beginning of
 the file.  The value returned may be subsequently passed to
 \haddockid{hSeek} to reposition the handle to the current position.
\par
This operation may fail with:
\par
\begin{itemize}
\item
 \haddocktt{isIllegalOperationError} if the Handle is not seekable.
\par

\end{itemize}

\end{haddockdesc}
\subsection{Handle properties
}
Each of these operations returns \haddockid{True} if the handle has the
 the specified property, or \haddockid{False} otherwise.
\par

\begin{haddockdesc}
\item[\begin{tabular}{@{}l}
hIsOpen\ ::\ Handle\ ->\ IO\ Bool\\hIsClosed\ ::\ Handle\ ->\ IO\ Bool\\hIsReadable\ ::\ Handle\ ->\ IO\ Bool\\hIsWritable\ ::\ Handle\ ->\ IO\ Bool\\hIsSeekable\ ::\ Handle\ ->\ IO\ Bool
\end{tabular}]
\end{haddockdesc}
\subsection{Terminal operations
}
\begin{haddockdesc}
\item[\begin{tabular}{@{}l}
hIsTerminalDevice\ ::\ Handle\ ->\ IO\ Bool
\end{tabular}]\haddockbegindoc
Is the handle connected to a terminal?
\par

\end{haddockdesc}
\begin{haddockdesc}
\item[\begin{tabular}{@{}l}
hSetEcho\ ::\ Handle\ ->\ Bool\ ->\ IO\ ()
\end{tabular}]\haddockbegindoc
Set the echoing status of a handle connected to a terminal.
\par

\end{haddockdesc}
\begin{haddockdesc}
\item[\begin{tabular}{@{}l}
hGetEcho\ ::\ Handle\ ->\ IO\ Bool
\end{tabular}]\haddockbegindoc
Get the echoing status of a handle connected to a terminal.
\par

\end{haddockdesc}
\subsection{Showing handle state
}
\begin{haddockdesc}
\item[\begin{tabular}{@{}l}
hShow\ ::\ Handle\ ->\ IO\ String
\end{tabular}]\haddockbegindoc
\haddockid{hShow} is in the \haddockid{IO} monad, and gives more comprehensive output
 than the (pure) instance of \haddockid{Show} for \haddockid{Handle}.
\par

\end{haddockdesc}
\section{Text input and output
}
\subsection{Text input
}
\begin{haddockdesc}
\item[\begin{tabular}{@{}l}
hWaitForInput\ ::\ Handle\ ->\ Int\ ->\ IO\ Bool
\end{tabular}]\haddockbegindoc
Computation \haddockid{hWaitForInput} \haddocktt{hdl\ t}
 waits until input is available on handle \haddocktt{hdl}.
 It returns \haddockid{True} as soon as input is available on \haddocktt{hdl},
 or \haddockid{False} if no input is available within \haddocktt{t} milliseconds.  Note that
 \haddockid{hWaitForInput} waits until one or more full \emph{characters} are available,
 which means that it needs to do decoding, and hence may fail
 with a decoding error.
\par
If \haddocktt{t} is less than zero, then \haddocktt{hWaitForInput} waits indefinitely.
\par
This operation may fail with:
\par
\begin{itemize}
\item
 \haddocktt{isEOFError} if the end of file has been reached.
\par

\item
 a decoding error, if the input begins with an invalid byte sequence
    in this Handle's encoding.
\par

\end{itemize}

\end{haddockdesc}
\begin{haddockdesc}
\item[\begin{tabular}{@{}l}
hReady\ ::\ Handle\ ->\ IO\ Bool
\end{tabular}]\haddockbegindoc
Computation \haddockid{hReady} \haddocktt{hdl} indicates whether at least one item is
 available for input from handle \haddocktt{hdl}.
\par
This operation may fail with:
\par
\begin{itemize}
\item
 \haddocktt{System.IO.Error.isEOFError} if the end of file has been reached.
\par

\end{itemize}

\end{haddockdesc}
\begin{haddockdesc}
\item[\begin{tabular}{@{}l}
hGetChar\ ::\ Handle\ ->\ IO\ Char
\end{tabular}]\haddockbegindoc
Computation \haddockid{hGetChar} \haddocktt{hdl} reads a character from the file or
 channel managed by \haddocktt{hdl}, blocking until a character is available.
\par
This operation may fail with:
\par
\begin{itemize}
\item
 \haddockid{isEOFError} if the end of file has been reached.
\par

\end{itemize}

\end{haddockdesc}
\begin{haddockdesc}
\item[\begin{tabular}{@{}l}
hGetLine\ ::\ Handle\ ->\ IO\ String
\end{tabular}]\haddockbegindoc
Computation \haddockid{hGetLine} \haddocktt{hdl} reads a line from the file or
 channel managed by \haddocktt{hdl}.
\par
This operation may fail with:
\par
\begin{itemize}
\item
 \haddockid{isEOFError} if the end of file is encountered when reading
    the \emph{first} character of the line.
\par

\end{itemize}
If \haddockid{hGetLine} encounters end-of-file at any other point while reading
 in a line, it is treated as a line terminator and the (partial)
 line is returned.
\par

\end{haddockdesc}
\begin{haddockdesc}
\item[\begin{tabular}{@{}l}
hLookAhead\ ::\ Handle\ ->\ IO\ Char
\end{tabular}]\haddockbegindoc
Computation \haddockid{hLookAhead} returns the next character from the handle
 without removing it from the input buffer, blocking until a character
 is available.
\par
This operation may fail with:
\par
\begin{itemize}
\item
 \haddocktt{isEOFError} if the end of file has been reached.
\par

\end{itemize}

\end{haddockdesc}
\begin{haddockdesc}
\item[\begin{tabular}{@{}l}
hGetContents\ ::\ Handle\ ->\ IO\ String
\end{tabular}]\haddockbegindoc
Computation \haddockid{hGetContents} \haddocktt{hdl} returns the list of characters
 corresponding to the unread portion of the channel or file managed
 by \haddocktt{hdl}, which is put into an intermediate state, \emph{semi-closed}.
 In this state, \haddocktt{hdl} is effectively closed,
 but items are read from \haddocktt{hdl} on demand and accumulated in a special
 list returned by \haddockid{hGetContents} \haddocktt{hdl}.
\par
Any operation that fails because a handle is closed,
 also fails if a handle is semi-closed.  The only exception is \haddocktt{hClose}.
 A semi-closed handle becomes closed:
\par
\begin{itemize}
\item
 if \haddocktt{hClose} is applied to it;
\par

\item
 if an I/O error occurs when reading an item from the handle;
\par

\item
 or once the entire contents of the handle has been read.
\par

\end{itemize}
Once a semi-closed handle becomes closed, the contents of the
 associated list becomes fixed.  The contents of this final list is
 only partially specified: it will contain at least all the items of
 the stream that were evaluated prior to the handle becoming closed.
\par
Any I/O errors encountered while a handle is semi-closed are simply
 discarded.
\par
This operation may fail with:
\par
\begin{itemize}
\item
 \haddockid{isEOFError} if the end of file has been reached.
\par

\end{itemize}

\end{haddockdesc}
\subsection{Text output
}
\begin{haddockdesc}
\item[\begin{tabular}{@{}l}
hPutChar\ ::\ Handle\ ->\ Char\ ->\ IO\ ()
\end{tabular}]\haddockbegindoc
Computation \haddockid{hPutChar} \haddocktt{hdl\ ch} writes the character \haddocktt{ch} to the
 file or channel managed by \haddocktt{hdl}.  Characters may be buffered if
 buffering is enabled for \haddocktt{hdl}.
\par
This operation may fail with:
\par
\begin{itemize}
\item
 \haddockid{isFullError} if the device is full; or
\par

\item
 \haddockid{isPermissionError} if another system resource limit would be exceeded.
\par

\end{itemize}

\end{haddockdesc}
\begin{haddockdesc}
\item[\begin{tabular}{@{}l}
hPutStr\ ::\ Handle\ ->\ String\ ->\ IO\ ()
\end{tabular}]\haddockbegindoc
Computation \haddockid{hPutStr} \haddocktt{hdl\ s} writes the string
 \haddocktt{s} to the file or channel managed by \haddocktt{hdl}.
\par
This operation may fail with:
\par
\begin{itemize}
\item
 \haddockid{isFullError} if the device is full; or
\par

\item
 \haddockid{isPermissionError} if another system resource limit would be exceeded.
\par

\end{itemize}

\end{haddockdesc}
\begin{haddockdesc}
\item[\begin{tabular}{@{}l}
hPutStrLn\ ::\ Handle\ ->\ String\ ->\ IO\ ()
\end{tabular}]\haddockbegindoc
The same as \haddockid{hPutStr}, but adds a newline character.
\par

\end{haddockdesc}
\begin{haddockdesc}
\item[\begin{tabular}{@{}l}
hPrint\ ::\ Show\ a\ =>\ Handle\ ->\ a\ ->\ IO\ ()
\end{tabular}]\haddockbegindoc
Computation \haddockid{hPrint} \haddocktt{hdl\ t} writes the string representation of \haddocktt{t}
 given by the \haddockid{shows} function to the file or channel managed by \haddocktt{hdl}
 and appends a newline.
\par
This operation may fail with:
\par
\begin{itemize}
\item
 \haddocktt{System.IO.Error.isFullError} if the device is full; or
\par

\item
 \haddocktt{System.IO.Error.isPermissionError} if another system resource limit would be exceeded.
\par

\end{itemize}

\end{haddockdesc}
\subsection{Special cases for standard input and output
}
These functions are also exported by the \haddocktt{Prelude}.
\par

\begin{haddockdesc}
\item[\begin{tabular}{@{}l}
interact\ ::\ (String\ ->\ String)\ ->\ IO\ ()
\end{tabular}]\haddockbegindoc
The \haddockid{interact} function takes a function of type \haddocktt{String->String}
 as its argument.  The entire input from the standard input device is
 passed to this function as its argument, and the resulting string is
 output on the standard output device.
\par

\end{haddockdesc}
\begin{haddockdesc}
\item[\begin{tabular}{@{}l}
putChar\ ::\ Char\ ->\ IO\ ()
\end{tabular}]\haddockbegindoc
Write a character to the standard output device
 (same as \haddockid{hPutChar} \haddockid{stdout}).
\par

\end{haddockdesc}
\begin{haddockdesc}
\item[\begin{tabular}{@{}l}
putStr\ ::\ String\ ->\ IO\ ()
\end{tabular}]\haddockbegindoc
Write a string to the standard output device
 (same as \haddockid{hPutStr} \haddockid{stdout}).
\par

\end{haddockdesc}
\begin{haddockdesc}
\item[\begin{tabular}{@{}l}
putStrLn\ ::\ String\ ->\ IO\ ()
\end{tabular}]\haddockbegindoc
The same as \haddockid{putStr}, but adds a newline character.
\par

\end{haddockdesc}
\begin{haddockdesc}
\item[\begin{tabular}{@{}l}
print\ ::\ Show\ a\ =>\ a\ ->\ IO\ ()
\end{tabular}]\haddockbegindoc
The \haddockid{print} function outputs a value of any printable type to the
 standard output device.
 Printable types are those that are instances of class \haddockid{Show}; \haddockid{print}
 converts values to strings for output using the \haddockid{show} operation and
 adds a newline.
\par
For example, a program to print the first 20 integers and their
 powers of 2 could be written as:
\par
\begin{quote}
{\haddockverb\begin{verbatim}
 main = print ([(n, 2^n) | n <- [0..19]])
\end{verbatim}}
\end{quote}

\end{haddockdesc}
\begin{haddockdesc}
\item[\begin{tabular}{@{}l}
getChar\ ::\ IO\ Char
\end{tabular}]\haddockbegindoc
Read a character from the standard input device
 (same as \haddockid{hGetChar} \haddockid{stdin}).
\par

\end{haddockdesc}
\begin{haddockdesc}
\item[\begin{tabular}{@{}l}
getLine\ ::\ IO\ String
\end{tabular}]\haddockbegindoc
Read a line from the standard input device
 (same as \haddockid{hGetLine} \haddockid{stdin}).
\par

\end{haddockdesc}
\begin{haddockdesc}
\item[\begin{tabular}{@{}l}
getContents\ ::\ IO\ String
\end{tabular}]\haddockbegindoc
The \haddockid{getContents} operation returns all user input as a single string,
 which is read lazily as it is needed
 (same as \haddockid{hGetContents} \haddockid{stdin}).
\par

\end{haddockdesc}
\begin{haddockdesc}
\item[\begin{tabular}{@{}l}
readIO\ ::\ Read\ a\ =>\ String\ ->\ IO\ a
\end{tabular}]\haddockbegindoc
The \haddockid{readIO} function is similar to \haddockid{read} except that it signals
 parse failure to the \haddockid{IO} monad instead of terminating the program.
\par

\end{haddockdesc}
\begin{haddockdesc}
\item[\begin{tabular}{@{}l}
readLn\ ::\ Read\ a\ =>\ IO\ a
\end{tabular}]\haddockbegindoc
The \haddockid{readLn} function combines \haddockid{getLine} and \haddockid{readIO}.
\par

\end{haddockdesc}