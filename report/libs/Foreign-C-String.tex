\haddockmoduleheading{Foreign.C.String}
\label{module:Foreign.C.String}
\haddockbeginheader
{\haddockverb\begin{verbatim}
module Foreign.C.String (
    CString,  CStringLen,  peekCString,  peekCStringLen,  newCString, 
    newCStringLen,  withCString,  withCStringLen,  charIsRepresentable, 
    castCharToCChar,  castCCharToChar,  castCharToCUChar,  castCUCharToChar, 
    castCharToCSChar,  castCSCharToChar,  peekCAString,  peekCAStringLen, 
    newCAString,  newCAStringLen,  withCAString,  withCAStringLen,  CWString, 
    CWStringLen,  peekCWString,  peekCWStringLen,  newCWString, 
    newCWStringLen,  withCWString,  withCWStringLen
  ) where\end{verbatim}}
\haddockendheader

Utilities for primitive marshalling of C strings.
\par
The marshalling converts each Haskell character, representing a Unicode
 code point, to one or more bytes in a manner that, by default, is
 determined by the current locale.  As a consequence, no guarantees
 can be made about the relative length of a Haskell string and its
 corresponding C string, and therefore all the marshalling routines
 include memory allocation.  The translation between Unicode and the
 encoding of the current locale may be lossy.
\par

\section{C strings
}
\begin{haddockdesc}
\item[\begin{tabular}{@{}l}
type\ CString\ =\ Ptr\ CChar
\end{tabular}]\haddockbegindoc
A C string is a reference to an array of C characters terminated by NUL.
\par

\end{haddockdesc}
\begin{haddockdesc}
\item[\begin{tabular}{@{}l}
type\ CStringLen\ =\ (Ptr\ CChar,\ Int)
\end{tabular}]\haddockbegindoc
A string with explicit length information in bytes instead of a
 terminating NUL (allowing NUL characters in the middle of the string).
\par

\end{haddockdesc}
\subsection{Using a locale-dependent encoding
}
Currently these functions are identical to their \haddocktt{CAString} counterparts;
 eventually they will use an encoding determined by the current locale.
\par

\begin{haddockdesc}
\item[\begin{tabular}{@{}l}
peekCString\ ::\ CString\ ->\ IO\ String
\end{tabular}]\haddockbegindoc
Marshal a NUL terminated C string into a Haskell string.
\par

\end{haddockdesc}
\begin{haddockdesc}
\item[\begin{tabular}{@{}l}
peekCStringLen\ ::\ CStringLen\ ->\ IO\ String
\end{tabular}]\haddockbegindoc
Marshal a C string with explicit length into a Haskell string.
\par

\end{haddockdesc}
\begin{haddockdesc}
\item[\begin{tabular}{@{}l}
newCString\ ::\ String\ ->\ IO\ CString
\end{tabular}]\haddockbegindoc
Marshal a Haskell string into a NUL terminated C string.
\par
\begin{itemize}
\item
 the Haskell string may \emph{not} contain any NUL characters
\par

\item
 new storage is allocated for the C string and must be
   explicitly freed using \haddocktt{Foreign.Marshal.Alloc.free} or
   \haddocktt{Foreign.Marshal.Alloc.finalizerFree}.
\par

\end{itemize}

\end{haddockdesc}
\begin{haddockdesc}
\item[\begin{tabular}{@{}l}
newCStringLen\ ::\ String\ ->\ IO\ CStringLen
\end{tabular}]\haddockbegindoc
Marshal a Haskell string into a C string (ie, character array) with
 explicit length information.
\par
\begin{itemize}
\item
 new storage is allocated for the C string and must be
   explicitly freed using \haddocktt{Foreign.Marshal.Alloc.free} or
   \haddocktt{Foreign.Marshal.Alloc.finalizerFree}.
\par

\end{itemize}

\end{haddockdesc}
\begin{haddockdesc}
\item[\begin{tabular}{@{}l}
withCString\ ::\ String\ ->\ (CString\ ->\ IO\ a)\ ->\ IO\ a
\end{tabular}]\haddockbegindoc
Marshal a Haskell string into a NUL terminated C string using temporary
 storage.
\par
\begin{itemize}
\item
 the Haskell string may \emph{not} contain any NUL characters
\par

\item
 the memory is freed when the subcomputation terminates (either
   normally or via an exception), so the pointer to the temporary
   storage must \emph{not} be used after this.
\par

\end{itemize}

\end{haddockdesc}
\begin{haddockdesc}
\item[\begin{tabular}{@{}l}
withCStringLen\ ::\ String\ ->\ (CStringLen\ ->\ IO\ a)\ ->\ IO\ a
\end{tabular}]\haddockbegindoc
Marshal a Haskell string into a C string (ie, character array)
 in temporary storage, with explicit length information.
\par
\begin{itemize}
\item
 the memory is freed when the subcomputation terminates (either
   normally or via an exception), so the pointer to the temporary
   storage must \emph{not} be used after this.
\par

\end{itemize}

\end{haddockdesc}
\begin{haddockdesc}
\item[\begin{tabular}{@{}l}
charIsRepresentable\ ::\ Char\ ->\ IO\ Bool
\end{tabular}]\haddockbegindoc
Determines whether a character can be accurately encoded in a \haddockid{CString}.
 Unrepresentable characters are converted to \haddocktt{'?'}.
\par
Currently only Latin-1 characters are representable.
\par

\end{haddockdesc}
\subsection{Using 8-bit characters
}
These variants of the above functions are for use with C libraries
 that are ignorant of Unicode.  These functions should be used with
 care, as a loss of information can occur.
\par

\begin{haddockdesc}
\item[\begin{tabular}{@{}l}
castCharToCChar\ ::\ Char\ ->\ CChar
\end{tabular}]\haddockbegindoc
Convert a Haskell character to a C character.
 This function is only safe on the first 256 characters.
\par

\end{haddockdesc}
\begin{haddockdesc}
\item[\begin{tabular}{@{}l}
castCCharToChar\ ::\ CChar\ ->\ Char
\end{tabular}]\haddockbegindoc
Convert a C byte, representing a Latin-1 character, to the corresponding
 Haskell character.
\par

\end{haddockdesc}
\begin{haddockdesc}
\item[\begin{tabular}{@{}l}
castCharToCUChar\ ::\ Char\ ->\ CUChar
\end{tabular}]\haddockbegindoc
Convert a Haskell character to a C \haddocktt{unsigned\ char}.
 This function is only safe on the first 256 characters.
\par

\end{haddockdesc}
\begin{haddockdesc}
\item[\begin{tabular}{@{}l}
castCUCharToChar\ ::\ CUChar\ ->\ Char
\end{tabular}]\haddockbegindoc
Convert a C \haddocktt{unsigned\ char}, representing a Latin-1 character, to
 the corresponding Haskell character.
\par

\end{haddockdesc}
\begin{haddockdesc}
\item[\begin{tabular}{@{}l}
castCharToCSChar\ ::\ Char\ ->\ CSChar
\end{tabular}]\haddockbegindoc
Convert a Haskell character to a C \haddocktt{signed\ char}.
 This function is only safe on the first 256 characters.
\par

\end{haddockdesc}
\begin{haddockdesc}
\item[\begin{tabular}{@{}l}
castCSCharToChar\ ::\ CSChar\ ->\ Char
\end{tabular}]\haddockbegindoc
Convert a C \haddocktt{signed\ char}, representing a Latin-1 character, to the
 corresponding Haskell character.
\par

\end{haddockdesc}
\begin{haddockdesc}
\item[\begin{tabular}{@{}l}
peekCAString\ ::\ CString\ ->\ IO\ String
\end{tabular}]\haddockbegindoc
Marshal a NUL terminated C string into a Haskell string.
\par

\end{haddockdesc}
\begin{haddockdesc}
\item[\begin{tabular}{@{}l}
peekCAStringLen\ ::\ CStringLen\ ->\ IO\ String
\end{tabular}]\haddockbegindoc
Marshal a C string with explicit length into a Haskell string.
\par

\end{haddockdesc}
\begin{haddockdesc}
\item[\begin{tabular}{@{}l}
newCAString\ ::\ String\ ->\ IO\ CString
\end{tabular}]\haddockbegindoc
Marshal a Haskell string into a NUL terminated C string.
\par
\begin{itemize}
\item
 the Haskell string may \emph{not} contain any NUL characters
\par

\item
 new storage is allocated for the C string and must be
   explicitly freed using \haddocktt{Foreign.Marshal.Alloc.free} or
   \haddocktt{Foreign.Marshal.Alloc.finalizerFree}.
\par

\end{itemize}

\end{haddockdesc}
\begin{haddockdesc}
\item[\begin{tabular}{@{}l}
newCAStringLen\ ::\ String\ ->\ IO\ CStringLen
\end{tabular}]\haddockbegindoc
Marshal a Haskell string into a C string (ie, character array) with
 explicit length information.
\par
\begin{itemize}
\item
 new storage is allocated for the C string and must be
   explicitly freed using \haddocktt{Foreign.Marshal.Alloc.free} or
   \haddocktt{Foreign.Marshal.Alloc.finalizerFree}.
\par

\end{itemize}

\end{haddockdesc}
\begin{haddockdesc}
\item[\begin{tabular}{@{}l}
withCAString\ ::\ String\ ->\ (CString\ ->\ IO\ a)\ ->\ IO\ a
\end{tabular}]\haddockbegindoc
Marshal a Haskell string into a NUL terminated C string using temporary
 storage.
\par
\begin{itemize}
\item
 the Haskell string may \emph{not} contain any NUL characters
\par

\item
 the memory is freed when the subcomputation terminates (either
   normally or via an exception), so the pointer to the temporary
   storage must \emph{not} be used after this.
\par

\end{itemize}

\end{haddockdesc}
\begin{haddockdesc}
\item[\begin{tabular}{@{}l}
withCAStringLen\ ::\ String\ ->\ (CStringLen\ ->\ IO\ a)\ ->\ IO\ a
\end{tabular}]\haddockbegindoc
Marshal a Haskell string into a C string (ie, character array)
 in temporary storage, with explicit length information.
\par
\begin{itemize}
\item
 the memory is freed when the subcomputation terminates (either
   normally or via an exception), so the pointer to the temporary
   storage must \emph{not} be used after this.
\par

\end{itemize}

\end{haddockdesc}
\section{C wide strings
}
These variants of the above functions are for use with C libraries
 that encode Unicode using the C \haddocktt{wchar{\char '137}t} type in a system-dependent
 way.  The only encodings supported are
\par
\begin{itemize}
\item
 UTF-32 (the C compiler defines \haddocktt{{\char '137}{\char '137}STDC{\char '137}ISO{\char '137}10646{\char '137}{\char '137}}), or
\par

\item
 UTF-16 (as used on Windows systems).
\par

\end{itemize}

\begin{haddockdesc}
\item[\begin{tabular}{@{}l}
type\ CWString\ =\ Ptr\ CWchar
\end{tabular}]\haddockbegindoc
A C wide string is a reference to an array of C wide characters
 terminated by NUL.
\par

\end{haddockdesc}
\begin{haddockdesc}
\item[\begin{tabular}{@{}l}
type\ CWStringLen\ =\ (Ptr\ CWchar,\ Int)
\end{tabular}]\haddockbegindoc
A wide character string with explicit length information in \haddockid{CWchar}s
 instead of a terminating NUL (allowing NUL characters in the middle
 of the string).
\par

\end{haddockdesc}
\begin{haddockdesc}
\item[\begin{tabular}{@{}l}
peekCWString\ ::\ CWString\ ->\ IO\ String
\end{tabular}]\haddockbegindoc
Marshal a NUL terminated C wide string into a Haskell string.
\par

\end{haddockdesc}
\begin{haddockdesc}
\item[\begin{tabular}{@{}l}
peekCWStringLen\ ::\ CWStringLen\ ->\ IO\ String
\end{tabular}]\haddockbegindoc
Marshal a C wide string with explicit length into a Haskell string.
\par

\end{haddockdesc}
\begin{haddockdesc}
\item[\begin{tabular}{@{}l}
newCWString\ ::\ String\ ->\ IO\ CWString
\end{tabular}]\haddockbegindoc
Marshal a Haskell string into a NUL terminated C wide string.
\par
\begin{itemize}
\item
 the Haskell string may \emph{not} contain any NUL characters
\par

\item
 new storage is allocated for the C wide string and must
   be explicitly freed using \haddocktt{Foreign.Marshal.Alloc.free} or
   \haddocktt{Foreign.Marshal.Alloc.finalizerFree}.
\par

\end{itemize}

\end{haddockdesc}
\begin{haddockdesc}
\item[\begin{tabular}{@{}l}
newCWStringLen\ ::\ String\ ->\ IO\ CWStringLen
\end{tabular}]\haddockbegindoc
Marshal a Haskell string into a C wide string (ie, wide character array)
 with explicit length information.
\par
\begin{itemize}
\item
 new storage is allocated for the C wide string and must
   be explicitly freed using \haddocktt{Foreign.Marshal.Alloc.free} or
   \haddocktt{Foreign.Marshal.Alloc.finalizerFree}.
\par

\end{itemize}

\end{haddockdesc}
\begin{haddockdesc}
\item[\begin{tabular}{@{}l}
withCWString\ ::\ String\ ->\ (CWString\ ->\ IO\ a)\ ->\ IO\ a
\end{tabular}]\haddockbegindoc
Marshal a Haskell string into a NUL terminated C wide string using
 temporary storage.
\par
\begin{itemize}
\item
 the Haskell string may \emph{not} contain any NUL characters
\par

\item
 the memory is freed when the subcomputation terminates (either
   normally or via an exception), so the pointer to the temporary
   storage must \emph{not} be used after this.
\par

\end{itemize}

\end{haddockdesc}
\begin{haddockdesc}
\item[\begin{tabular}{@{}l}
withCWStringLen\ ::\ String\ ->\ (CWStringLen\ ->\ IO\ a)\ ->\ IO\ a
\end{tabular}]\haddockbegindoc
Marshal a Haskell string into a NUL terminated C wide string using
 temporary storage.
\par
\begin{itemize}
\item
 the Haskell string may \emph{not} contain any NUL characters
\par

\item
 the memory is freed when the subcomputation terminates (either
   normally or via an exception), so the pointer to the temporary
   storage must \emph{not} be used after this.
\par

\end{itemize}

\end{haddockdesc}