\haddockmoduleheading{Data.Array}
\label{module:Data.Array}
\haddockbeginheader
{\haddockverb\begin{verbatim}
module Data.Array (
    module Data.Ix,  Array,  array,  listArray,  accumArray,  (!),  bounds, 
    indices,  elems,  assocs,  (//),  accum,  ixmap
  ) where\end{verbatim}}
\haddockendheader

\section{Immutable non-strict arrays
}
Haskell provides indexable \emph{arrays}, which may be thought of as functions
whose domains are isomorphic to contiguous subsets of the integers.
Functions restricted in this way can be implemented efficiently;
in particular, a programmer may reasonably expect rapid access to
the components.  To ensure the possibility of such an implementation,
arrays are treated as data, not as general functions.
\par
Since most array functions involve the class \haddockid{Ix}, the contents of the
module \haddocktt{Data.Ix} are re-exported from \haddocktt{Data.Array} for convenience:
\par

\begin{haddockdesc}
\item[\begin{tabular}{@{}l}
module\ Data.Ix
\end{tabular}]
\end{haddockdesc}
\begin{haddockdesc}
\item[\begin{tabular}{@{}l}
data\ Ix\ i\ =>\ Array\ i\ e
\end{tabular}]\haddockbegindoc
The type of immutable non-strict (boxed) arrays
 with indices in \haddocktt{i} and elements in \haddocktt{e}.
\par

\end{haddockdesc}
\begin{haddockdesc}
\item[\begin{tabular}{@{}l}
instance\ Ix\ i\ =>\ Functor\ (Array\ i)\\instance\ (Ix\ i,\ Eq\ e)\ =>\ Eq\ (Array\ i\ e)\\instance\ (Ix\ i,\ Ord\ e)\ =>\ Ord\ (Array\ i\ e)\\instance\ (Ix\ a,\ Read\ a,\ Read\ b)\ =>\ Read\ (Array\ a\ b)\\instance\ (Ix\ a,\ Show\ a,\ Show\ b)\ =>\ Show\ (Array\ a\ b)
\end{tabular}]
\end{haddockdesc}
\section{Array construction
}
\begin{haddockdesc}
\item[\begin{tabular}{@{}l}
array
\end{tabular}]\haddockbegindoc
\haddockbeginargs
\haddockdecltt{::} & \haddockdecltt{Ix i} \\
                     \haddockdecltt{=>} & \haddockdecltt{(i, i)} & a pair of \emph{bounds}, each of the index type
 of the array.  These bounds are the lowest and
 highest indices in the array, in that order.
 For example, a one-origin vector of length
 '10' has bounds '(1,10)', and a one-origin '10'
 by '10' matrix has bounds '((1,1),(10,10))'.
 \\
                                                                   \haddockdecltt{->} & \haddockdecltt{[(i, e)]} & a list of \emph{associations} of the form
 (\emph{index}, \emph{value}).  Typically, this list will
 be expressed as a comprehension.  An
 association '(i, x)' defines the value of
 the array at index \haddocktt{i} to be \haddocktt{x}.
 \\
                                                                                                                   \haddockdecltt{->} & \haddockdecltt{Array i e} & \\
\end{tabulary}\par
Construct an array with the specified bounds and containing values
 for given indices within these bounds.
\par
The array is undefined (i.e. bottom) if any index in the list is
 out of bounds.  If any
 two associations in the list have the same index, the value at that
 index is undefined (i.e. bottom).
\par
Because the indices must be checked for these errors, \haddockid{array} is
 strict in the bounds argument and in the indices of the association
 list, but non-strict in the values.  Thus, recurrences such as the
 following are possible:
\par
\begin{quote}
{\haddockverb\begin{verbatim}
 a = array (1,100) ((1,1) : [(i, i * a!(i-1)) | i <- [2..100]])
\end{verbatim}}
\end{quote}
Not every index within the bounds of the array need appear in the
 association list, but the values associated with indices that do not
 appear will be undefined (i.e. bottom).
\par
If, in any dimension, the lower bound is greater than the upper bound,
 then the array is legal, but empty.  Indexing an empty array always
 gives an array-bounds error, but \haddockid{bounds} still yields the bounds
 with which the array was constructed.
\par

\end{haddockdesc}
\begin{haddockdesc}
\item[\begin{tabular}{@{}l}
listArray\ ::\ Ix\ i\ =>\ (i,\ i)\ ->\ {\char 91}e{\char 93}\ ->\ Array\ i\ e
\end{tabular}]\haddockbegindoc
Construct an array from a pair of bounds and a list of values in
 index order.
\par

\end{haddockdesc}
\begin{haddockdesc}
\item[\begin{tabular}{@{}l}
accumArray
\end{tabular}]\haddockbegindoc
\haddockbeginargs
\haddockdecltt{::} & \haddockdecltt{Ix i} \\
                     \haddockdecltt{=>} & \haddockdecltt{(e
                                                          -> a
                                                             -> e)} & accumulating function
 \\
                                                                      \haddockdecltt{->} & \haddockdecltt{e} & initial value
 \\
                                                                                                               \haddockdecltt{->} & \haddockdecltt{(i, i)} & bounds of the array
 \\
                                                                                                                                                             \haddockdecltt{->} & \haddockdecltt{[(i, a)]} & association list
 \\
                                                                                                                                                                                                             \haddockdecltt{->} & \haddockdecltt{Array i e} & \\
\end{tabulary}\par
The \haddockid{accumArray} function deals with repeated indices in the association
 list using an \emph{accumulating function} which combines the values of
 associations with the same index.
 For example, given a list of values of some index type, \haddocktt{hist}
 produces a histogram of the number of occurrences of each index within
 a specified range:
\par
\begin{quote}
{\haddockverb\begin{verbatim}
 hist :: (Ix a, Num b) => (a,a) -> [a] -> Array a b
 hist bnds is = accumArray (+) 0 bnds [(i, 1) | i<-is, inRange bnds i]
\end{verbatim}}
\end{quote}
If the accumulating function is strict, then \haddockid{accumArray} is strict in
 the values, as well as the indices, in the association list.  Thus,
 unlike ordinary arrays built with \haddockid{array}, accumulated arrays should
 not in general be recursive.
\par

\end{haddockdesc}
\section{Accessing arrays
}
\begin{haddockdesc}
\item[\begin{tabular}{@{}l}
(!)\ ::\ Ix\ i\ =>\ Array\ i\ e\ ->\ i\ ->\ e
\end{tabular}]\haddockbegindoc
The value at the given index in an array.
\par

\end{haddockdesc}
\begin{haddockdesc}
\item[\begin{tabular}{@{}l}
bounds\ ::\ Ix\ i\ =>\ Array\ i\ e\ ->\ (i,\ i)
\end{tabular}]\haddockbegindoc
The bounds with which an array was constructed.
\par

\end{haddockdesc}
\begin{haddockdesc}
\item[\begin{tabular}{@{}l}
indices\ ::\ Ix\ i\ =>\ Array\ i\ e\ ->\ {\char 91}i{\char 93}
\end{tabular}]\haddockbegindoc
The list of indices of an array in ascending order.
\par

\end{haddockdesc}
\begin{haddockdesc}
\item[\begin{tabular}{@{}l}
elems\ ::\ Ix\ i\ =>\ Array\ i\ e\ ->\ {\char 91}e{\char 93}
\end{tabular}]\haddockbegindoc
The list of elements of an array in index order.
\par

\end{haddockdesc}
\begin{haddockdesc}
\item[\begin{tabular}{@{}l}
assocs\ ::\ Ix\ i\ =>\ Array\ i\ e\ ->\ {\char 91}(i,\ e){\char 93}
\end{tabular}]\haddockbegindoc
The list of associations of an array in index order.
\par

\end{haddockdesc}
\section{Incremental array updates
}
\begin{haddockdesc}
\item[\begin{tabular}{@{}l}
(//)\ ::\ Ix\ i\ =>\ Array\ i\ e\ ->\ {\char 91}(i,\ e){\char 93}\ ->\ Array\ i\ e
\end{tabular}]\haddockbegindoc
Constructs an array identical to the first argument except that it has
 been updated by the associations in the right argument.
 For example, if \haddocktt{m} is a 1-origin, \haddocktt{n} by \haddocktt{n} matrix, then
\par
\begin{quote}
{\haddockverb\begin{verbatim}
 m//[((i,i), 0) | i <- [1..n]]
\end{verbatim}}
\end{quote}
is the same matrix, except with the diagonal zeroed.
\par
Repeated indices in the association list are handled as for \haddockid{array}:
 the resulting array is undefined (i.e. bottom),
\par

\end{haddockdesc}
\begin{haddockdesc}
\item[\begin{tabular}{@{}l}
accum\ ::\ Ix\ i\ =>\ (e\ ->\ a\ ->\ e)\\\ \ \ \ \ \ \ \ \ \ \ \ \ \ \ \ \ ->\ Array\ i\ e\ ->\ {\char 91}(i,\ a){\char 93}\ ->\ Array\ i\ e
\end{tabular}]\haddockbegindoc
\haddocktt{accum\ f} takes an array and an association list and accumulates
 pairs from the list into the array with the accumulating function \haddocktt{f}.
 Thus \haddockid{accumArray} can be defined using \haddockid{accum}:
\par
\begin{quote}
{\haddockverb\begin{verbatim}
 accumArray f z b = accum f (array b [(i, z) | i <- range b])
\end{verbatim}}
\end{quote}

\end{haddockdesc}
\section{Derived arrays
}
\begin{haddockdesc}
\item[\begin{tabular}{@{}l}
ixmap\ ::\ (Ix\ i,\ Ix\ j)\ =>\ (i,\ i)\\\ \ \ \ \ \ \ \ \ \ \ \ \ \ \ \ \ \ \ \ \ \ \ \ \ ->\ (i\ ->\ j)\ ->\ Array\ j\ e\ ->\ Array\ i\ e
\end{tabular}]\haddockbegindoc
\haddockid{ixmap} allows for transformations on array indices.
 It may be thought of as providing function composition on the right
 with the mapping that the original array embodies.
\par
A similar transformation of array values may be achieved using \haddockid{fmap}
 from the \haddockid{Array} instance of the \haddockid{Functor} class.
\par

\end{haddockdesc}
\section{Specification
}
\begin{quote}
{\haddockverb\begin{verbatim}
 module  Array ( 
     module Data.Ix,  -- export all of Data.Ix
     Array, array, listArray, (!), bounds, indices, elems, assocs, 
     accumArray, (//), accum, ixmap ) where
 
 import Data.Ix
 import Data.List( (\\) )
 
 infixl 9  !, //
 
 data (Ix a) => Array a b = MkArray (a,a) (a -> b) deriving ()
 
 array       :: (Ix a) => (a,a) -> [(a,b)] -> Array a b
 array b ivs
   | any (not . inRange b. fst) ivs
      = error "Data.Array.array: out-of-range array association"
   | otherwise
      = MkArray b arr
   where
     arr j = case [ v | (i,v) <- ivs, i == j ] of
               [v]   -> v
               []    -> error "Data.Array.!: undefined array element"
               _     -> error "Data.Array.!: multiply defined array element"
 
 listArray             :: (Ix a) => (a,a) -> [b] -> Array a b
 listArray b vs        =  array b (zipWith (\ a b -> (a,b)) (range b) vs)
 
 (!)                   :: (Ix a) => Array a b -> a -> b
 (!) (MkArray _ f)     =  f
 
 bounds                :: (Ix a) => Array a b -> (a,a)
 bounds (MkArray b _)  =  b
 
 indices               :: (Ix a) => Array a b -> [a]
 indices               =  range . bounds
 
 elems                 :: (Ix a) => Array a b -> [b]
 elems a               =  [a!i | i <- indices a]
 
 assocs                :: (Ix a) => Array a b -> [(a,b)]
 assocs a              =  [(i, a!i) | i <- indices a]
 
 (//)                  :: (Ix a) => Array a b -> [(a,b)] -> Array a b
 a // new_ivs          = array (bounds a) (old_ivs ++ new_ivs)
                       where
                         old_ivs = [(i,a!i) | i <- indices a,
                                              i `notElem` new_is]
                         new_is  = [i | (i,_) <- new_ivs]
 
 accum                 :: (Ix a) => (b -> c -> b) -> Array a b -> [(a,c)]
                                    -> Array a b
 accum f               =  foldl (\a (i,v) -> a // [(i,f (a!i) v)])
 
 accumArray            :: (Ix a) => (b -> c -> b) -> b -> (a,a) -> [(a,c)]
                                    -> Array a b
 accumArray f z b      =  accum f (array b [(i,z) | i <- range b])
 
 ixmap                 :: (Ix a, Ix b) => (a,a) -> (a -> b) -> Array b c
                                          -> Array a c
 ixmap b f a           = array b [(i, a ! f i) | i <- range b]
 
 instance  (Ix a)          => Functor (Array a) where
     fmap fn (MkArray b f) =  MkArray b (fn . f) 
 
 instance  (Ix a, Eq b)  => Eq (Array a b)  where
     a == a' =  assocs a == assocs a'
 
 instance  (Ix a, Ord b) => Ord (Array a b)  where
     a <= a' =  assocs a <= assocs a'
 
 instance  (Ix a, Show a, Show b) => Show (Array a b)  where
     showsPrec p a = showParen (p > arrPrec) (
                     showString "array " .
                     showsPrec (arrPrec+1) (bounds a) . showChar ' ' .
                     showsPrec (arrPrec+1) (assocs a)                  )
 
 instance  (Ix a, Read a, Read b) => Read (Array a b)  where
     readsPrec p = readParen (p > arrPrec)
            (\r -> [ (array b as, u) 
                   | ("array",s) <- lex r,
                     (b,t)       <- readsPrec (arrPrec+1) s,
                     (as,u)      <- readsPrec (arrPrec+1) t ])
 
 -- Precedence of the 'array' function is that of application itself
 arrPrec = 10
\end{verbatim}}
\end{quote}
