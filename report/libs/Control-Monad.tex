\haddockmoduleheading{Control.Monad}
\label{module:Control.Monad}
\haddockbeginheader
{\haddockverb\begin{verbatim}
module Control.Monad (
    Functor(fmap),  Monad((>>=), (>>), return, fail),  MonadPlus(mzero, mplus), 
    mapM,  mapM_,  forM,  forM_,  sequence,  sequence_,  (=<<),  (>=>),  (<=<), 
    forever,  void,  join,  msum,  filterM,  mapAndUnzipM,  zipWithM, 
    zipWithM_,  foldM,  foldM_,  replicateM,  replicateM_,  guard,  when, 
    unless,  liftM,  liftM2,  liftM3,  liftM4,  liftM5,  ap
  ) where\end{verbatim}}
\haddockendheader

The \haddocktt{Control.Monad} module provides the \haddockid{Functor}, \haddockid{Monad} and
 \haddockid{MonadPlus} classes, together with some useful operations on monads.
\par

\section{Functor and monad classes
}
\begin{haddockdesc}
\item[\begin{tabular}{@{}l}
class\ Functor\ f\ where
\end{tabular}]\haddockbegindoc
The \haddockid{Functor} class is used for types that can be mapped over.
Instances of \haddockid{Functor} should satisfy the following laws:
\par
\begin{quote}
{\haddockverb\begin{verbatim}
 fmap id  ==  id
 fmap (f . g)  ==  fmap f . fmap g
\end{verbatim}}
\end{quote}
The instances of \haddockid{Functor} for lists, \haddocktt{Data.Maybe.Maybe} and \haddocktt{System.IO.IO}
satisfy these laws.
\par

\haddockpremethods{}\textbf{Methods}
\begin{haddockdesc}
\item[\begin{tabular}{@{}l}
fmap\ ::\ (a\ ->\ b)\ ->\ f\ a\ ->\ f\ b
\end{tabular}]
\end{haddockdesc}
\end{haddockdesc}
\begin{haddockdesc}
\item[\begin{tabular}{@{}l}
instance\ Functor\ {\char 91}{\char 93}\\instance\ Functor\ IO\\instance\ Functor\ Maybe\\instance\ Ix\ i\ =>\ Functor\ (Array\ i)
\end{tabular}]
\end{haddockdesc}
\begin{haddockdesc}
\item[\begin{tabular}{@{}l}
class\ Monad\ m\ where
\end{tabular}]\haddockbegindoc
The \haddockid{Monad} class defines the basic operations over a \emph{monad},
a concept from a branch of mathematics known as \emph{category theory}.
From the perspective of a Haskell programmer, however, it is best to
think of a monad as an \emph{abstract datatype} of actions.
Haskell's \haddocktt{do} expressions provide a convenient syntax for writing
monadic expressions.
\par
Minimal complete definition: \haddockid{>>=} and \haddockid{return}.
\par
Instances of \haddockid{Monad} should satisfy the following laws:
\par
\begin{quote}
{\haddockverb\begin{verbatim}
 return a >>= k  ==  k a
 m >>= return  ==  m
 m >>= (\x -> k x >>= h)  ==  (m >>= k) >>= h
\end{verbatim}}
\end{quote}
Instances of both \haddockid{Monad} and \haddockid{Functor} should additionally satisfy the law:
\par
\begin{quote}
{\haddockverb\begin{verbatim}
 fmap f xs  ==  xs >>= return . f
\end{verbatim}}
\end{quote}
The instances of \haddockid{Monad} for lists, \haddocktt{Data.Maybe.Maybe} and \haddocktt{System.IO.IO}
defined in the \haddocktt{Prelude} satisfy these laws.
\par

\haddockpremethods{}\textbf{Methods}
\begin{haddockdesc}
\item[\begin{tabular}{@{}l}
(>>=)\ ::\ m\ a\ ->\ (a\ ->\ m\ b)\ ->\ m\ b
\end{tabular}]\haddockbegindoc
Sequentially compose two actions, passing any value produced
 by the first as an argument to the second.
\par

\end{haddockdesc}
\begin{haddockdesc}
\item[\begin{tabular}{@{}l}
(>>)\ ::\ m\ a\ ->\ m\ b\ ->\ m\ b
\end{tabular}]\haddockbegindoc
Sequentially compose two actions, discarding any value produced
 by the first, like sequencing operators (such as the semicolon)
 in imperative languages.
\par

\end{haddockdesc}
\begin{haddockdesc}
\item[\begin{tabular}{@{}l}
return\ ::\ a\ ->\ m\ a
\end{tabular}]\haddockbegindoc
Inject a value into the monadic type.
\par

\end{haddockdesc}
\begin{haddockdesc}
\item[\begin{tabular}{@{}l}
fail\ ::\ String\ ->\ m\ a
\end{tabular}]\haddockbegindoc
Fail with a message.  This operation is not part of the
 mathematical definition of a monad, but is invoked on pattern-match
 failure in a \haddocktt{do} expression.
\par

\end{haddockdesc}
\end{haddockdesc}
\begin{haddockdesc}
\item[\begin{tabular}{@{}l}
instance\ Monad\ {\char 91}{\char 93}\\instance\ Monad\ IO\\instance\ Monad\ Maybe
\end{tabular}]
\end{haddockdesc}
\begin{haddockdesc}
\item[\begin{tabular}{@{}l}
class\ Monad\ m\ =>\ MonadPlus\ m\ where
\end{tabular}]\haddockbegindoc
Monads that also support choice and failure.
\par

\haddockpremethods{}\textbf{Methods}
\begin{haddockdesc}
\item[\begin{tabular}{@{}l}
mzero\ ::\ m\ a
\end{tabular}]\haddockbegindoc
the identity of \haddockid{mplus}.  It should also satisfy the equations
\par
\begin{quote}
{\haddockverb\begin{verbatim}
 mzero >>= f  =  mzero
 v >> mzero   =  mzero
\end{verbatim}}
\end{quote}

\end{haddockdesc}
\begin{haddockdesc}
\item[\begin{tabular}{@{}l}
mplus\ ::\ m\ a\ ->\ m\ a\ ->\ m\ a
\end{tabular}]\haddockbegindoc
an associative operation
\par

\end{haddockdesc}
\end{haddockdesc}
\begin{haddockdesc}
\item[\begin{tabular}{@{}l}
instance\ MonadPlus\ {\char 91}{\char 93}\\instance\ MonadPlus\ Maybe
\end{tabular}]
\end{haddockdesc}
\section{Functions
}
\subsection{Naming conventions
}
The functions in this library use the following naming conventions: 
\par
\begin{itemize}
\item
 A postfix '\haddocktt{M}' always stands for a function in the Kleisli category:
  The monad type constructor \haddocktt{m} is added to function results
  (modulo currying) and nowhere else.  So, for example, 
\par

\end{itemize}
\begin{quote}
{\haddockverb\begin{verbatim}
  filter  ::              (a ->   Bool) -> [a] ->   [a]
  filterM :: (Monad m) => (a -> m Bool) -> [a] -> m [a]
\end{verbatim}}
\end{quote}
\begin{itemize}
\item
 A postfix '\haddocktt{{\char '137}}' changes the result type from \haddocktt{(m\ a)} to \haddocktt{(m\ ())}.
  Thus, for example: 
\par

\end{itemize}
\begin{quote}
{\haddockverb\begin{verbatim}
  sequence  :: Monad m => [m a] -> m [a] 
  sequence_ :: Monad m => [m a] -> m () 
\end{verbatim}}
\end{quote}
\begin{itemize}
\item
 A prefix '\haddocktt{m}' generalizes an existing function to a monadic form.
  Thus, for example: 
\par

\end{itemize}
\begin{quote}
{\haddockverb\begin{verbatim}
  sum  :: Num a       => [a]   -> a
  msum :: MonadPlus m => [m a] -> m a
\end{verbatim}}
\end{quote}

\subsection{Basic \haddocktt{Monad} functions
}
\begin{haddockdesc}
\item[\begin{tabular}{@{}l}
mapM\ ::\ Monad\ m\ =>\ (a\ ->\ m\ b)\ ->\ {\char 91}a{\char 93}\ ->\ m\ {\char 91}b{\char 93}
\end{tabular}]\haddockbegindoc
\haddocktt{mapM\ f} is equivalent to \haddocktt{sequence\ .\ map\ f}.
\par

\end{haddockdesc}
\begin{haddockdesc}
\item[\begin{tabular}{@{}l}
mapM{\char '137}\ ::\ Monad\ m\ =>\ (a\ ->\ m\ b)\ ->\ {\char 91}a{\char 93}\ ->\ m\ ()
\end{tabular}]\haddockbegindoc
\haddocktt{mapM{\char '137}\ f} is equivalent to \haddocktt{sequence{\char '137}\ .\ map\ f}.
\par

\end{haddockdesc}
\begin{haddockdesc}
\item[\begin{tabular}{@{}l}
forM\ ::\ Monad\ m\ =>\ {\char 91}a{\char 93}\ ->\ (a\ ->\ m\ b)\ ->\ m\ {\char 91}b{\char 93}
\end{tabular}]\haddockbegindoc
\haddockid{forM} is \haddockid{mapM} with its arguments flipped
\par

\end{haddockdesc}
\begin{haddockdesc}
\item[\begin{tabular}{@{}l}
forM{\char '137}\ ::\ Monad\ m\ =>\ {\char 91}a{\char 93}\ ->\ (a\ ->\ m\ b)\ ->\ m\ ()
\end{tabular}]\haddockbegindoc
\haddockid{forM{\char '137}} is \haddockid{mapM{\char '137}} with its arguments flipped
\par

\end{haddockdesc}
\begin{haddockdesc}
\item[\begin{tabular}{@{}l}
sequence\ ::\ Monad\ m\ =>\ {\char 91}m\ a{\char 93}\ ->\ m\ {\char 91}a{\char 93}
\end{tabular}]\haddockbegindoc
Evaluate each action in the sequence from left to right,
 and collect the results.
\par

\end{haddockdesc}
\begin{haddockdesc}
\item[\begin{tabular}{@{}l}
sequence{\char '137}\ ::\ Monad\ m\ =>\ {\char 91}m\ a{\char 93}\ ->\ m\ ()
\end{tabular}]\haddockbegindoc
Evaluate each action in the sequence from left to right,
 and ignore the results.
\par

\end{haddockdesc}
\begin{haddockdesc}
\item[\begin{tabular}{@{}l}
(=<<)\ ::\ Monad\ m\ =>\ (a\ ->\ m\ b)\ ->\ m\ a\ ->\ m\ b
\end{tabular}]\haddockbegindoc
Same as \haddockid{>>=}, but with the arguments interchanged.
\par

\end{haddockdesc}
\begin{haddockdesc}
\item[\begin{tabular}{@{}l}
(>=>)\ ::\ Monad\ m\ =>\ (a\ ->\ m\ b)\ ->\ (b\ ->\ m\ c)\ ->\ a\ ->\ m\ c
\end{tabular}]\haddockbegindoc
Left-to-right Kleisli composition of monads.
\par

\end{haddockdesc}
\begin{haddockdesc}
\item[\begin{tabular}{@{}l}
(<=<)\ ::\ Monad\ m\ =>\ (b\ ->\ m\ c)\ ->\ (a\ ->\ m\ b)\ ->\ a\ ->\ m\ c
\end{tabular}]\haddockbegindoc
Right-to-left Kleisli composition of monads. \haddocktt{(>=>)}, with the arguments flipped
\par

\end{haddockdesc}
\begin{haddockdesc}
\item[\begin{tabular}{@{}l}
forever\ ::\ Monad\ m\ =>\ m\ a\ ->\ m\ b
\end{tabular}]\haddockbegindoc
\haddocktt{forever\ act} repeats the action infinitely.
\par

\end{haddockdesc}
\begin{haddockdesc}
\item[\begin{tabular}{@{}l}
void\ ::\ Functor\ f\ =>\ f\ a\ ->\ f\ ()
\end{tabular}]\haddockbegindoc
\haddocktt{void\ value} discards or ignores the result of evaluation, such as the return value of an \haddockid{IO} action.
\par

\end{haddockdesc}
\subsection{Generalisations of list functions
}
\begin{haddockdesc}
\item[\begin{tabular}{@{}l}
join\ ::\ Monad\ m\ =>\ m\ (m\ a)\ ->\ m\ a
\end{tabular}]\haddockbegindoc
The \haddockid{join} function is the conventional monad join operator. It is used to
 remove one level of monadic structure, projecting its bound argument into the
 outer level.
\par

\end{haddockdesc}
\begin{haddockdesc}
\item[\begin{tabular}{@{}l}
msum\ ::\ MonadPlus\ m\ =>\ {\char 91}m\ a{\char 93}\ ->\ m\ a
\end{tabular}]\haddockbegindoc
This generalizes the list-based \haddockid{concat} function.
\par

\end{haddockdesc}
\begin{haddockdesc}
\item[\begin{tabular}{@{}l}
filterM\ ::\ Monad\ m\ =>\ (a\ ->\ m\ Bool)\ ->\ {\char 91}a{\char 93}\ ->\ m\ {\char 91}a{\char 93}
\end{tabular}]\haddockbegindoc
This generalizes the list-based \haddockid{filter} function.
\par

\end{haddockdesc}
\begin{haddockdesc}
\item[\begin{tabular}{@{}l}
mapAndUnzipM\ ::\ Monad\ m\ =>\ (a\ ->\ m\ (b,\ c))\ ->\ {\char 91}a{\char 93}\ ->\ m\ ({\char 91}b{\char 93},\ {\char 91}c{\char 93})
\end{tabular}]\haddockbegindoc
The \haddockid{mapAndUnzipM} function maps its first argument over a list, returning
 the result as a pair of lists. This function is mainly used with complicated
 data structures or a state-transforming monad.
\par

\end{haddockdesc}
\begin{haddockdesc}
\item[\begin{tabular}{@{}l}
zipWithM\ ::\ Monad\ m\ =>\ (a\ ->\ b\ ->\ m\ c)\ ->\ {\char 91}a{\char 93}\ ->\ {\char 91}b{\char 93}\ ->\ m\ {\char 91}c{\char 93}
\end{tabular}]\haddockbegindoc
The \haddockid{zipWithM} function generalizes \haddockid{zipWith} to arbitrary monads.
\par

\end{haddockdesc}
\begin{haddockdesc}
\item[\begin{tabular}{@{}l}
zipWithM{\char '137}\ ::\ Monad\ m\ =>\ (a\ ->\ b\ ->\ m\ c)\ ->\ {\char 91}a{\char 93}\ ->\ {\char 91}b{\char 93}\ ->\ m\ ()
\end{tabular}]\haddockbegindoc
\haddockid{zipWithM{\char '137}} is the extension of \haddockid{zipWithM} which ignores the final result.
\par

\end{haddockdesc}
\begin{haddockdesc}
\item[\begin{tabular}{@{}l}
foldM\ ::\ Monad\ m\ =>\ (a\ ->\ b\ ->\ m\ a)\ ->\ a\ ->\ {\char 91}b{\char 93}\ ->\ m\ a
\end{tabular}]\haddockbegindoc
The \haddockid{foldM} function is analogous to \haddockid{foldl}, except that its result is
encapsulated in a monad. Note that \haddockid{foldM} works from left-to-right over
the list arguments. This could be an issue where \haddocktt{(>>)} and the `folded
function' are not commutative.
\par
\begin{quote}
{\haddockverb\begin{verbatim}
       foldM f a1 [x1, x2, ..., xm]
\end{verbatim}}
\end{quote}
==  
\par
\begin{quote}
{\haddockverb\begin{verbatim}
       do
         a2 <- f a1 x1
         a3 <- f a2 x2
         ...
         f am xm
\end{verbatim}}
\end{quote}
If right-to-left evaluation is required, the input list should be reversed.
\par

\end{haddockdesc}
\begin{haddockdesc}
\item[\begin{tabular}{@{}l}
foldM{\char '137}\ ::\ Monad\ m\ =>\ (a\ ->\ b\ ->\ m\ a)\ ->\ a\ ->\ {\char 91}b{\char 93}\ ->\ m\ ()
\end{tabular}]\haddockbegindoc
Like \haddockid{foldM}, but discards the result.
\par

\end{haddockdesc}
\begin{haddockdesc}
\item[\begin{tabular}{@{}l}
replicateM\ ::\ Monad\ m\ =>\ Int\ ->\ m\ a\ ->\ m\ {\char 91}a{\char 93}
\end{tabular}]\haddockbegindoc
\haddocktt{replicateM\ n\ act} performs the action \haddocktt{n} times,
 gathering the results.
\par

\end{haddockdesc}
\begin{haddockdesc}
\item[\begin{tabular}{@{}l}
replicateM{\char '137}\ ::\ Monad\ m\ =>\ Int\ ->\ m\ a\ ->\ m\ ()
\end{tabular}]\haddockbegindoc
Like \haddockid{replicateM}, but discards the result.
\par

\end{haddockdesc}
\subsection{Conditional execution of monadic expressions
}
\begin{haddockdesc}
\item[\begin{tabular}{@{}l}
guard\ ::\ MonadPlus\ m\ =>\ Bool\ ->\ m\ ()
\end{tabular}]\haddockbegindoc
\haddocktt{guard\ b} is \haddocktt{return\ ()} if \haddocktt{b} is \haddockid{True},
 and \haddockid{mzero} if \haddocktt{b} is \haddockid{False}.
\par

\end{haddockdesc}
\begin{haddockdesc}
\item[\begin{tabular}{@{}l}
when\ ::\ Monad\ m\ =>\ Bool\ ->\ m\ ()\ ->\ m\ ()
\end{tabular}]\haddockbegindoc
Conditional execution of monadic expressions. For example, 
\par
\begin{quote}
{\haddockverb\begin{verbatim}
       when debug (putStr "Debugging\n")
\end{verbatim}}
\end{quote}
will output the string \haddocktt{Debugging{\char '134}n} if the Boolean value \haddocktt{debug} is \haddockid{True},
and otherwise do nothing.
\par

\end{haddockdesc}
\begin{haddockdesc}
\item[\begin{tabular}{@{}l}
unless\ ::\ Monad\ m\ =>\ Bool\ ->\ m\ ()\ ->\ m\ ()
\end{tabular}]\haddockbegindoc
The reverse of \haddockid{when}.
\par

\end{haddockdesc}
\subsection{Monadic lifting operators
}
\begin{haddockdesc}
\item[\begin{tabular}{@{}l}
liftM\ ::\ Monad\ m\ =>\ (a1\ ->\ r)\ ->\ m\ a1\ ->\ m\ r
\end{tabular}]\haddockbegindoc
Promote a function to a monad.
\par

\end{haddockdesc}
\begin{haddockdesc}
\item[\begin{tabular}{@{}l}
liftM2\ ::\ Monad\ m\ =>\ (a1\ ->\ a2\ ->\ r)\ ->\ m\ a1\ ->\ m\ a2\ ->\ m\ r
\end{tabular}]\haddockbegindoc
Promote a function to a monad, scanning the monadic arguments from
 left to right.  For example,
\par
\begin{quote}
{\haddockverb\begin{verbatim}
    liftM2 (+) [0,1] [0,2] = [0,2,1,3]
    liftM2 (+) (Just 1) Nothing = Nothing
\end{verbatim}}
\end{quote}

\end{haddockdesc}
\begin{haddockdesc}
\item[\begin{tabular}{@{}l}
liftM3\ ::\ Monad\ m\ =>\ (a1\ ->\ a2\ ->\ a3\ ->\ r)\\\ \ \ \ \ \ \ \ \ \ \ \ \ \ \ \ \ \ \ \ \ ->\ m\ a1\ ->\ m\ a2\ ->\ m\ a3\ ->\ m\ r
\end{tabular}]\haddockbegindoc
Promote a function to a monad, scanning the monadic arguments from
 left to right (cf. \haddockid{liftM2}).
\par

\end{haddockdesc}
\begin{haddockdesc}
\item[\begin{tabular}{@{}l}
liftM4\ ::\ Monad\ m\ =>\ (a1\ ->\ a2\ ->\ a3\ ->\ a4\ ->\ r)\\\ \ \ \ \ \ \ \ \ \ \ \ \ \ \ \ \ \ \ \ \ ->\ m\ a1\ ->\ m\ a2\ ->\ m\ a3\ ->\ m\ a4\ ->\ m\ r
\end{tabular}]\haddockbegindoc
Promote a function to a monad, scanning the monadic arguments from
 left to right (cf. \haddockid{liftM2}).
\par

\end{haddockdesc}
\begin{haddockdesc}
\item[\begin{tabular}{@{}l}
liftM5\ ::\ Monad\ m\ =>\ (a1\ ->\ a2\ ->\ a3\ ->\ a4\ ->\ a5\ ->\ r)\\\ \ \ \ \ \ \ \ \ \ \ \ \ \ \ \ \ \ \ \ \ ->\ m\ a1\ ->\ m\ a2\ ->\ m\ a3\ ->\ m\ a4\ ->\ m\ a5\ ->\ m\ r
\end{tabular}]\haddockbegindoc
Promote a function to a monad, scanning the monadic arguments from
 left to right (cf. \haddockid{liftM2}).
\par

\end{haddockdesc}
\begin{haddockdesc}
\item[\begin{tabular}{@{}l}
ap\ ::\ Monad\ m\ =>\ m\ (a\ ->\ b)\ ->\ m\ a\ ->\ m\ b
\end{tabular}]\haddockbegindoc
In many situations, the \haddockid{liftM} operations can be replaced by uses of
\haddockid{ap}, which promotes function application. 
\par
\begin{quote}
{\haddockverb\begin{verbatim}
       return f `ap` x1 `ap` ... `ap` xn
\end{verbatim}}
\end{quote}
is equivalent to 
\par
\begin{quote}
{\haddockverb\begin{verbatim}
       liftMn f x1 x2 ... xn
\end{verbatim}}
\end{quote}

\end{haddockdesc}