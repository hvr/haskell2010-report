\haddockmoduleheading{Data.List}
\label{module:Data.List}
\haddockbeginheader
{\haddockverb\begin{verbatim}
module Data.List (
    (++),  head,  last,  tail,  init,  null,  length,  map,  reverse, 
    intersperse,  intercalate,  transpose,  subsequences,  permutations, 
    foldl,  foldl',  foldl1,  foldl1',  foldr,  foldr1,  concat,  concatMap, 
    and,  or,  any,  all,  sum,  product,  maximum,  minimum,  scanl,  scanl1, 
    scanr,  scanr1,  mapAccumL,  mapAccumR,  iterate,  repeat,  replicate, 
    cycle,  unfoldr,  take,  drop,  splitAt,  takeWhile,  dropWhile,  span, 
    break,  stripPrefix,  group,  inits,  tails,  isPrefixOf,  isSuffixOf, 
    isInfixOf,  elem,  notElem,  lookup,  find,  filter,  partition,  (!!), 
    elemIndex,  elemIndices,  findIndex,  findIndices,  zip,  zip3,  zip4, 
    zip5,  zip6,  zip7,  zipWith,  zipWith3,  zipWith4,  zipWith5,  zipWith6, 
    zipWith7,  unzip,  unzip3,  unzip4,  unzip5,  unzip6,  unzip7,  lines, 
    words,  unlines,  unwords,  nub,  delete,  (\\),  union,  intersect,  sort, 
    insert,  nubBy,  deleteBy,  deleteFirstsBy,  unionBy,  intersectBy, 
    groupBy,  sortBy,  insertBy,  maximumBy,  minimumBy,  genericLength, 
    genericTake,  genericDrop,  genericSplitAt,  genericIndex,  genericReplicate
  ) where\end{verbatim}}
\haddockendheader

\section{Basic functions
}
\begin{haddockdesc}
\item[\begin{tabular}{@{}l}
(++)\ ::\ {\char 91}a{\char 93}\ ->\ {\char 91}a{\char 93}\ ->\ {\char 91}a{\char 93}
\end{tabular}]\haddockbegindoc
Append two lists, i.e.,
\par
\begin{quote}
{\haddockverb\begin{verbatim}
 [x1, ..., xm] ++ [y1, ..., yn] == [x1, ..., xm, y1, ..., yn]
 [x1, ..., xm] ++ [y1, ...] == [x1, ..., xm, y1, ...]
\end{verbatim}}
\end{quote}
If the first list is not finite, the result is the first list.
\par

\end{haddockdesc}
\begin{haddockdesc}
\item[\begin{tabular}{@{}l}
head\ ::\ {\char 91}a{\char 93}\ ->\ a
\end{tabular}]\haddockbegindoc
Extract the first element of a list, which must be non-empty.
\par

\end{haddockdesc}
\begin{haddockdesc}
\item[\begin{tabular}{@{}l}
last\ ::\ {\char 91}a{\char 93}\ ->\ a
\end{tabular}]\haddockbegindoc
Extract the last element of a list, which must be finite and non-empty.
\par

\end{haddockdesc}
\begin{haddockdesc}
\item[\begin{tabular}{@{}l}
tail\ ::\ {\char 91}a{\char 93}\ ->\ {\char 91}a{\char 93}
\end{tabular}]\haddockbegindoc
Extract the elements after the head of a list, which must be non-empty.
\par

\end{haddockdesc}
\begin{haddockdesc}
\item[\begin{tabular}{@{}l}
init\ ::\ {\char 91}a{\char 93}\ ->\ {\char 91}a{\char 93}
\end{tabular}]\haddockbegindoc
Return all the elements of a list except the last one.
 The list must be non-empty.
\par

\end{haddockdesc}
\begin{haddockdesc}
\item[\begin{tabular}{@{}l}
null\ ::\ {\char 91}a{\char 93}\ ->\ Bool
\end{tabular}]\haddockbegindoc
Test whether a list is empty.
\par

\end{haddockdesc}
\begin{haddockdesc}
\item[\begin{tabular}{@{}l}
length\ ::\ {\char 91}a{\char 93}\ ->\ Int
\end{tabular}]\haddockbegindoc
\emph{O(n)}. \haddockid{length} returns the length of a finite list as an \haddockid{Int}.
 It is an instance of the more general \haddocktt{Data.List.genericLength},
 the result type of which may be any kind of number.
\par

\end{haddockdesc}
\section{List transformations
}
\begin{haddockdesc}
\item[\begin{tabular}{@{}l}
map\ ::\ (a\ ->\ b)\ ->\ {\char 91}a{\char 93}\ ->\ {\char 91}b{\char 93}
\end{tabular}]\haddockbegindoc
\haddockid{map} \haddocktt{f\ xs} is the list obtained by applying \haddocktt{f} to each element
 of \haddocktt{xs}, i.e.,
\par
\begin{quote}
{\haddockverb\begin{verbatim}
 map f [x1, x2, ..., xn] == [f x1, f x2, ..., f xn]
 map f [x1, x2, ...] == [f x1, f x2, ...]
\end{verbatim}}
\end{quote}

\end{haddockdesc}
\begin{haddockdesc}
\item[\begin{tabular}{@{}l}
reverse\ ::\ {\char 91}a{\char 93}\ ->\ {\char 91}a{\char 93}
\end{tabular}]\haddockbegindoc
\haddockid{reverse} \haddocktt{xs} returns the elements of \haddocktt{xs} in reverse order.
 \haddocktt{xs} must be finite.
\par

\end{haddockdesc}
\begin{haddockdesc}
\item[\begin{tabular}{@{}l}
intersperse\ ::\ a\ ->\ {\char 91}a{\char 93}\ ->\ {\char 91}a{\char 93}
\end{tabular}]\haddockbegindoc
The \haddockid{intersperse} function takes an element and a list and
 `intersperses' that element between the elements of the list.
 For example,
\par
\begin{quote}
{\haddockverb\begin{verbatim}
 intersperse ',' "abcde" == "a,b,c,d,e"
\end{verbatim}}
\end{quote}

\end{haddockdesc}
\begin{haddockdesc}
\item[\begin{tabular}{@{}l}
intercalate\ ::\ {\char 91}a{\char 93}\ ->\ {\char 91}{\char 91}a{\char 93}{\char 93}\ ->\ {\char 91}a{\char 93}
\end{tabular}]\haddockbegindoc
\haddockid{intercalate} \haddocktt{xs\ xss} is equivalent to \haddocktt{(concat\ (intersperse\ xs\ xss))}.
 It inserts the list \haddocktt{xs} in between the lists in \haddocktt{xss} and concatenates the
 result.
\par

\end{haddockdesc}
\begin{haddockdesc}
\item[\begin{tabular}{@{}l}
transpose\ ::\ {\char 91}{\char 91}a{\char 93}{\char 93}\ ->\ {\char 91}{\char 91}a{\char 93}{\char 93}
\end{tabular}]\haddockbegindoc
The \haddockid{transpose} function transposes the rows and columns of its argument.
 For example,
\par
\begin{quote}
{\haddockverb\begin{verbatim}
 transpose [[1,2,3],[4,5,6]] == [[1,4],[2,5],[3,6]]
\end{verbatim}}
\end{quote}

\end{haddockdesc}
\begin{haddockdesc}
\item[\begin{tabular}{@{}l}
subsequences\ ::\ {\char 91}a{\char 93}\ ->\ {\char 91}{\char 91}a{\char 93}{\char 93}
\end{tabular}]\haddockbegindoc
The \haddockid{subsequences} function returns the list of all subsequences of the argument.
\par
\begin{quote}
{\haddockverb\begin{verbatim}
 subsequences "abc" == ["","a","b","ab","c","ac","bc","abc"]
\end{verbatim}}
\end{quote}

\end{haddockdesc}
\begin{haddockdesc}
\item[\begin{tabular}{@{}l}
permutations\ ::\ {\char 91}a{\char 93}\ ->\ {\char 91}{\char 91}a{\char 93}{\char 93}
\end{tabular}]\haddockbegindoc
The \haddockid{permutations} function returns the list of all permutations of the argument.
\par
\begin{quote}
{\haddockverb\begin{verbatim}
 permutations "abc" == ["abc","bac","cba","bca","cab","acb"]
\end{verbatim}}
\end{quote}

\end{haddockdesc}
\section{Reducing lists (folds)
}
\begin{haddockdesc}
\item[\begin{tabular}{@{}l}
foldl\ ::\ (a\ ->\ b\ ->\ a)\ ->\ a\ ->\ {\char 91}b{\char 93}\ ->\ a
\end{tabular}]\haddockbegindoc
\haddockid{foldl}, applied to a binary operator, a starting value (typically
 the left-identity of the operator), and a list, reduces the list
 using the binary operator, from left to right:
\par
\begin{quote}
{\haddockverb\begin{verbatim}
 foldl f z [x1, x2, ..., xn] == (...((z `f` x1) `f` x2) `f`...) `f` xn
\end{verbatim}}
\end{quote}
The list must be finite.
\par

\end{haddockdesc}
\begin{haddockdesc}
\item[\begin{tabular}{@{}l}
foldl'\ ::\ (a\ ->\ b\ ->\ a)\ ->\ a\ ->\ {\char 91}b{\char 93}\ ->\ a
\end{tabular}]\haddockbegindoc
A strict version of \haddockid{foldl}.
\par

\end{haddockdesc}
\begin{haddockdesc}
\item[\begin{tabular}{@{}l}
foldl1\ ::\ (a\ ->\ a\ ->\ a)\ ->\ {\char 91}a{\char 93}\ ->\ a
\end{tabular}]\haddockbegindoc
\haddockid{foldl1} is a variant of \haddockid{foldl} that has no starting value argument,
 and thus must be applied to non-empty lists.
\par

\end{haddockdesc}
\begin{haddockdesc}
\item[\begin{tabular}{@{}l}
foldl1'\ ::\ (a\ ->\ a\ ->\ a)\ ->\ {\char 91}a{\char 93}\ ->\ a
\end{tabular}]\haddockbegindoc
A strict version of \haddockid{foldl1}
\par

\end{haddockdesc}
\begin{haddockdesc}
\item[\begin{tabular}{@{}l}
foldr\ ::\ (a\ ->\ b\ ->\ b)\ ->\ b\ ->\ {\char 91}a{\char 93}\ ->\ b
\end{tabular}]\haddockbegindoc
\haddockid{foldr}, applied to a binary operator, a starting value (typically
 the right-identity of the operator), and a list, reduces the list
 using the binary operator, from right to left:
\par
\begin{quote}
{\haddockverb\begin{verbatim}
 foldr f z [x1, x2, ..., xn] == x1 `f` (x2 `f` ... (xn `f` z)...)
\end{verbatim}}
\end{quote}

\end{haddockdesc}
\begin{haddockdesc}
\item[\begin{tabular}{@{}l}
foldr1\ ::\ (a\ ->\ a\ ->\ a)\ ->\ {\char 91}a{\char 93}\ ->\ a
\end{tabular}]\haddockbegindoc
\haddockid{foldr1} is a variant of \haddockid{foldr} that has no starting value argument,
 and thus must be applied to non-empty lists.
\par

\end{haddockdesc}
\subsection{Special folds
}
\begin{haddockdesc}
\item[\begin{tabular}{@{}l}
concat\ ::\ {\char 91}{\char 91}a{\char 93}{\char 93}\ ->\ {\char 91}a{\char 93}
\end{tabular}]\haddockbegindoc
Concatenate a list of lists.
\par

\end{haddockdesc}
\begin{haddockdesc}
\item[\begin{tabular}{@{}l}
concatMap\ ::\ (a\ ->\ {\char 91}b{\char 93})\ ->\ {\char 91}a{\char 93}\ ->\ {\char 91}b{\char 93}
\end{tabular}]\haddockbegindoc
Map a function over a list and concatenate the results.
\par

\end{haddockdesc}
\begin{haddockdesc}
\item[\begin{tabular}{@{}l}
and\ ::\ {\char 91}Bool{\char 93}\ ->\ Bool
\end{tabular}]\haddockbegindoc
\haddockid{and} returns the conjunction of a Boolean list.  For the result to be
 \haddockid{True}, the list must be finite; \haddockid{False}, however, results from a \haddockid{False}
 value at a finite index of a finite or infinite list.
\par

\end{haddockdesc}
\begin{haddockdesc}
\item[\begin{tabular}{@{}l}
or\ ::\ {\char 91}Bool{\char 93}\ ->\ Bool
\end{tabular}]\haddockbegindoc
\haddockid{or} returns the disjunction of a Boolean list.  For the result to be
 \haddockid{False}, the list must be finite; \haddockid{True}, however, results from a \haddockid{True}
 value at a finite index of a finite or infinite list.
\par

\end{haddockdesc}
\begin{haddockdesc}
\item[\begin{tabular}{@{}l}
any\ ::\ (a\ ->\ Bool)\ ->\ {\char 91}a{\char 93}\ ->\ Bool
\end{tabular}]\haddockbegindoc
Applied to a predicate and a list, \haddockid{any} determines if any element
 of the list satisfies the predicate.
\par

\end{haddockdesc}
\begin{haddockdesc}
\item[\begin{tabular}{@{}l}
all\ ::\ (a\ ->\ Bool)\ ->\ {\char 91}a{\char 93}\ ->\ Bool
\end{tabular}]\haddockbegindoc
Applied to a predicate and a list, \haddockid{all} determines if all elements
 of the list satisfy the predicate.
\par

\end{haddockdesc}
\begin{haddockdesc}
\item[\begin{tabular}{@{}l}
sum\ ::\ Num\ a\ =>\ {\char 91}a{\char 93}\ ->\ a
\end{tabular}]\haddockbegindoc
The \haddockid{sum} function computes the sum of a finite list of numbers.
\par

\end{haddockdesc}
\begin{haddockdesc}
\item[\begin{tabular}{@{}l}
product\ ::\ Num\ a\ =>\ {\char 91}a{\char 93}\ ->\ a
\end{tabular}]\haddockbegindoc
The \haddockid{product} function computes the product of a finite list of numbers.
\par

\end{haddockdesc}
\begin{haddockdesc}
\item[\begin{tabular}{@{}l}
maximum\ ::\ Ord\ a\ =>\ {\char 91}a{\char 93}\ ->\ a
\end{tabular}]\haddockbegindoc
\haddockid{maximum} returns the maximum value from a list,
 which must be non-empty, finite, and of an ordered type.
 It is a special case of \haddockid{maximumBy}, which allows the
 programmer to supply their own comparison function.
\par

\end{haddockdesc}
\begin{haddockdesc}
\item[\begin{tabular}{@{}l}
minimum\ ::\ Ord\ a\ =>\ {\char 91}a{\char 93}\ ->\ a
\end{tabular}]\haddockbegindoc
\haddockid{minimum} returns the minimum value from a list,
 which must be non-empty, finite, and of an ordered type.
 It is a special case of \haddockid{minimumBy}, which allows the
 programmer to supply their own comparison function.
\par

\end{haddockdesc}
\section{Building lists
}
\subsection{Scans
}
\begin{haddockdesc}
\item[\begin{tabular}{@{}l}
scanl\ ::\ (a\ ->\ b\ ->\ a)\ ->\ a\ ->\ {\char 91}b{\char 93}\ ->\ {\char 91}a{\char 93}
\end{tabular}]\haddockbegindoc
\haddockid{scanl} is similar to \haddockid{foldl}, but returns a list of successive
 reduced values from the left:
\par
\begin{quote}
{\haddockverb\begin{verbatim}
 scanl f z [x1, x2, ...] == [z, z `f` x1, (z `f` x1) `f` x2, ...]
\end{verbatim}}
\end{quote}
Note that
\par
\begin{quote}
{\haddockverb\begin{verbatim}
 last (scanl f z xs) == foldl f z xs.
\end{verbatim}}
\end{quote}

\end{haddockdesc}
\begin{haddockdesc}
\item[\begin{tabular}{@{}l}
scanl1\ ::\ (a\ ->\ a\ ->\ a)\ ->\ {\char 91}a{\char 93}\ ->\ {\char 91}a{\char 93}
\end{tabular}]\haddockbegindoc
\haddockid{scanl1} is a variant of \haddockid{scanl} that has no starting value argument:
\par
\begin{quote}
{\haddockverb\begin{verbatim}
 scanl1 f [x1, x2, ...] == [x1, x1 `f` x2, ...]
\end{verbatim}}
\end{quote}

\end{haddockdesc}
\begin{haddockdesc}
\item[\begin{tabular}{@{}l}
scanr\ ::\ (a\ ->\ b\ ->\ b)\ ->\ b\ ->\ {\char 91}a{\char 93}\ ->\ {\char 91}b{\char 93}
\end{tabular}]\haddockbegindoc
\haddockid{scanr} is the right-to-left dual of \haddockid{scanl}.
 Note that
\par
\begin{quote}
{\haddockverb\begin{verbatim}
 head (scanr f z xs) == foldr f z xs.
\end{verbatim}}
\end{quote}

\end{haddockdesc}
\begin{haddockdesc}
\item[\begin{tabular}{@{}l}
scanr1\ ::\ (a\ ->\ a\ ->\ a)\ ->\ {\char 91}a{\char 93}\ ->\ {\char 91}a{\char 93}
\end{tabular}]\haddockbegindoc
\haddockid{scanr1} is a variant of \haddockid{scanr} that has no starting value argument.
\par

\end{haddockdesc}
\subsection{Accumulating maps
}
\begin{haddockdesc}
\item[\begin{tabular}{@{}l}
mapAccumL\ ::\ (acc\ ->\ x\ ->\ (acc,\ y))\ ->\ acc\ ->\ {\char 91}x{\char 93}\ ->\ (acc,\ {\char 91}y{\char 93})
\end{tabular}]\haddockbegindoc
The \haddockid{mapAccumL} function behaves like a combination of \haddockid{map} and
 \haddockid{foldl}; it applies a function to each element of a list, passing
 an accumulating parameter from left to right, and returning a final
 value of this accumulator together with the new list.
\par

\end{haddockdesc}
\begin{haddockdesc}
\item[\begin{tabular}{@{}l}
mapAccumR\ ::\ (acc\ ->\ x\ ->\ (acc,\ y))\ ->\ acc\ ->\ {\char 91}x{\char 93}\ ->\ (acc,\ {\char 91}y{\char 93})
\end{tabular}]\haddockbegindoc
The \haddockid{mapAccumR} function behaves like a combination of \haddockid{map} and
 \haddockid{foldr}; it applies a function to each element of a list, passing
 an accumulating parameter from right to left, and returning a final
 value of this accumulator together with the new list.
\par

\end{haddockdesc}
\subsection{Infinite lists
}
\begin{haddockdesc}
\item[\begin{tabular}{@{}l}
iterate\ ::\ (a\ ->\ a)\ ->\ a\ ->\ {\char 91}a{\char 93}
\end{tabular}]\haddockbegindoc
\haddockid{iterate} \haddocktt{f\ x} returns an infinite list of repeated applications
 of \haddocktt{f} to \haddocktt{x}:
\par
\begin{quote}
{\haddockverb\begin{verbatim}
 iterate f x == [x, f x, f (f x), ...]
\end{verbatim}}
\end{quote}

\end{haddockdesc}
\begin{haddockdesc}
\item[\begin{tabular}{@{}l}
repeat\ ::\ a\ ->\ {\char 91}a{\char 93}
\end{tabular}]\haddockbegindoc
\haddockid{repeat} \haddocktt{x} is an infinite list, with \haddocktt{x} the value of every element.
\par

\end{haddockdesc}
\begin{haddockdesc}
\item[\begin{tabular}{@{}l}
replicate\ ::\ Int\ ->\ a\ ->\ {\char 91}a{\char 93}
\end{tabular}]\haddockbegindoc
\haddockid{replicate} \haddocktt{n\ x} is a list of length \haddocktt{n} with \haddocktt{x} the value of
 every element.
 It is an instance of the more general \haddocktt{Data.List.genericReplicate},
 in which \haddocktt{n} may be of any integral type.
\par

\end{haddockdesc}
\begin{haddockdesc}
\item[\begin{tabular}{@{}l}
cycle\ ::\ {\char 91}a{\char 93}\ ->\ {\char 91}a{\char 93}
\end{tabular}]\haddockbegindoc
\haddockid{cycle} ties a finite list into a circular one, or equivalently,
 the infinite repetition of the original list.  It is the identity
 on infinite lists.
\par

\end{haddockdesc}
\subsection{Unfolding
}
\begin{haddockdesc}
\item[\begin{tabular}{@{}l}
unfoldr\ ::\ (b\ ->\ Maybe\ (a,\ b))\ ->\ b\ ->\ {\char 91}a{\char 93}
\end{tabular}]\haddockbegindoc
The \haddockid{unfoldr} function is a `dual' to \haddockid{foldr}: while \haddockid{foldr}
 reduces a list to a summary value, \haddockid{unfoldr} builds a list from
 a seed value.  The function takes the element and returns \haddockid{Nothing}
 if it is done producing the list or returns \haddockid{Just} \haddocktt{(a,b)}, in which
 case, \haddocktt{a} is a prepended to the list and \haddocktt{b} is used as the next
 element in a recursive call.  For example,
\par
\begin{quote}
{\haddockverb\begin{verbatim}
 iterate f == unfoldr (\x -> Just (x, f x))
\end{verbatim}}
\end{quote}
In some cases, \haddockid{unfoldr} can undo a \haddockid{foldr} operation:
\par
\begin{quote}
{\haddockverb\begin{verbatim}
 unfoldr f' (foldr f z xs) == xs
\end{verbatim}}
\end{quote}
if the following holds:
\par
\begin{quote}
{\haddockverb\begin{verbatim}
 f' (f x y) = Just (x,y)
 f' z       = Nothing
\end{verbatim}}
\end{quote}
A simple use of unfoldr:
\par
\begin{quote}
{\haddockverb\begin{verbatim}
 unfoldr (\b -> if b == 0 then Nothing else Just (b, b-1)) 10
  [10,9,8,7,6,5,4,3,2,1]
\end{verbatim}}
\end{quote}

\end{haddockdesc}
\section{Sublists
}
\subsection{Extracting sublists
}
\begin{haddockdesc}
\item[\begin{tabular}{@{}l}
take\ ::\ Int\ ->\ {\char 91}a{\char 93}\ ->\ {\char 91}a{\char 93}
\end{tabular}]\haddockbegindoc
\haddockid{take} \haddocktt{n}, applied to a list \haddocktt{xs}, returns the prefix of \haddocktt{xs}
 of length \haddocktt{n}, or \haddocktt{xs} itself if \haddocktt{n\ >\ length\ xs}:
\par
\begin{quote}
{\haddockverb\begin{verbatim}
 take 5 "Hello World!" == "Hello"
 take 3 [1,2,3,4,5] == [1,2,3]
 take 3 [1,2] == [1,2]
 take 3 [] == []
 take (-1) [1,2] == []
 take 0 [1,2] == []
\end{verbatim}}
\end{quote}
It is an instance of the more general \haddocktt{Data.List.genericTake},
 in which \haddocktt{n} may be of any integral type.
\par

\end{haddockdesc}
\begin{haddockdesc}
\item[\begin{tabular}{@{}l}
drop\ ::\ Int\ ->\ {\char 91}a{\char 93}\ ->\ {\char 91}a{\char 93}
\end{tabular}]\haddockbegindoc
\haddockid{drop} \haddocktt{n\ xs} returns the suffix of \haddocktt{xs}
 after the first \haddocktt{n} elements, or \haddocktt{{\char 91}{\char 93}} if \haddocktt{n\ >\ length\ xs}:
\par
\begin{quote}
{\haddockverb\begin{verbatim}
 drop 6 "Hello World!" == "World!"
 drop 3 [1,2,3,4,5] == [4,5]
 drop 3 [1,2] == []
 drop 3 [] == []
 drop (-1) [1,2] == [1,2]
 drop 0 [1,2] == [1,2]
\end{verbatim}}
\end{quote}
It is an instance of the more general \haddocktt{Data.List.genericDrop},
 in which \haddocktt{n} may be of any integral type.
\par

\end{haddockdesc}
\begin{haddockdesc}
\item[\begin{tabular}{@{}l}
splitAt\ ::\ Int\ ->\ {\char 91}a{\char 93}\ ->\ ({\char 91}a{\char 93},\ {\char 91}a{\char 93})
\end{tabular}]\haddockbegindoc
\haddockid{splitAt} \haddocktt{n\ xs} returns a tuple where first element is \haddocktt{xs} prefix of
 length \haddocktt{n} and second element is the remainder of the list:
\par
\begin{quote}
{\haddockverb\begin{verbatim}
 splitAt 6 "Hello World!" == ("Hello ","World!")
 splitAt 3 [1,2,3,4,5] == ([1,2,3],[4,5])
 splitAt 1 [1,2,3] == ([1],[2,3])
 splitAt 3 [1,2,3] == ([1,2,3],[])
 splitAt 4 [1,2,3] == ([1,2,3],[])
 splitAt 0 [1,2,3] == ([],[1,2,3])
 splitAt (-1) [1,2,3] == ([],[1,2,3])
\end{verbatim}}
\end{quote}
It is equivalent to \haddocktt{(take\ n\ xs,\ drop\ n\ xs)}.
 \haddockid{splitAt} is an instance of the more general \haddocktt{Data.List.genericSplitAt},
 in which \haddocktt{n} may be of any integral type.
\par

\end{haddockdesc}
\begin{haddockdesc}
\item[\begin{tabular}{@{}l}
takeWhile\ ::\ (a\ ->\ Bool)\ ->\ {\char 91}a{\char 93}\ ->\ {\char 91}a{\char 93}
\end{tabular}]\haddockbegindoc
\haddockid{takeWhile}, applied to a predicate \haddocktt{p} and a list \haddocktt{xs}, returns the
 longest prefix (possibly empty) of \haddocktt{xs} of elements that satisfy \haddocktt{p}:
\par
\begin{quote}
{\haddockverb\begin{verbatim}
 takeWhile (< 3) [1,2,3,4,1,2,3,4] == [1,2]
 takeWhile (< 9) [1,2,3] == [1,2,3]
 takeWhile (< 0) [1,2,3] == []
\end{verbatim}}
\end{quote}

\end{haddockdesc}
\begin{haddockdesc}
\item[\begin{tabular}{@{}l}
dropWhile\ ::\ (a\ ->\ Bool)\ ->\ {\char 91}a{\char 93}\ ->\ {\char 91}a{\char 93}
\end{tabular}]\haddockbegindoc
\haddockid{dropWhile} \haddocktt{p\ xs} returns the suffix remaining after \haddockid{takeWhile} \haddocktt{p\ xs}:
\par
\begin{quote}
{\haddockverb\begin{verbatim}
 dropWhile (< 3) [1,2,3,4,5,1,2,3] == [3,4,5,1,2,3]
 dropWhile (< 9) [1,2,3] == []
 dropWhile (< 0) [1,2,3] == [1,2,3]
\end{verbatim}}
\end{quote}

\end{haddockdesc}
\begin{haddockdesc}
\item[\begin{tabular}{@{}l}
span\ ::\ (a\ ->\ Bool)\ ->\ {\char 91}a{\char 93}\ ->\ ({\char 91}a{\char 93},\ {\char 91}a{\char 93})
\end{tabular}]\haddockbegindoc
\haddockid{span}, applied to a predicate \haddocktt{p} and a list \haddocktt{xs}, returns a tuple where
 first element is longest prefix (possibly empty) of \haddocktt{xs} of elements that
 satisfy \haddocktt{p} and second element is the remainder of the list:
\par
\begin{quote}
{\haddockverb\begin{verbatim}
 span (< 3) [1,2,3,4,1,2,3,4] == ([1,2],[3,4,1,2,3,4])
 span (< 9) [1,2,3] == ([1,2,3],[])
 span (< 0) [1,2,3] == ([],[1,2,3])
\end{verbatim}}
\end{quote}
\haddockid{span} \haddocktt{p\ xs} is equivalent to \haddocktt{(takeWhile\ p\ xs,\ dropWhile\ p\ xs)}
\par

\end{haddockdesc}
\begin{haddockdesc}
\item[\begin{tabular}{@{}l}
break\ ::\ (a\ ->\ Bool)\ ->\ {\char 91}a{\char 93}\ ->\ ({\char 91}a{\char 93},\ {\char 91}a{\char 93})
\end{tabular}]\haddockbegindoc
\haddockid{break}, applied to a predicate \haddocktt{p} and a list \haddocktt{xs}, returns a tuple where
 first element is longest prefix (possibly empty) of \haddocktt{xs} of elements that
 \emph{do not satisfy} \haddocktt{p} and second element is the remainder of the list:
\par
\begin{quote}
{\haddockverb\begin{verbatim}
 break (> 3) [1,2,3,4,1,2,3,4] == ([1,2,3],[4,1,2,3,4])
 break (< 9) [1,2,3] == ([],[1,2,3])
 break (> 9) [1,2,3] == ([1,2,3],[])
\end{verbatim}}
\end{quote}
\haddockid{break} \haddocktt{p} is equivalent to \haddocktt{span\ (not\ .\ p)}.
\par

\end{haddockdesc}
\begin{haddockdesc}
\item[\begin{tabular}{@{}l}
stripPrefix\ ::\ Eq\ a\ =>\ {\char 91}a{\char 93}\ ->\ {\char 91}a{\char 93}\ ->\ Maybe\ {\char 91}a{\char 93}
\end{tabular}]\haddockbegindoc
The \haddockid{stripPrefix} function drops the given prefix from a list.
 It returns \haddockid{Nothing} if the list did not start with the prefix
 given, or \haddockid{Just} the list after the prefix, if it does.
\par
\begin{quote}
{\haddockverb\begin{verbatim}
 stripPrefix "foo" "foobar" == Just "bar"
 stripPrefix "foo" "foo" == Just ""
 stripPrefix "foo" "barfoo" == Nothing
 stripPrefix "foo" "barfoobaz" == Nothing
\end{verbatim}}
\end{quote}

\end{haddockdesc}
\begin{haddockdesc}
\item[\begin{tabular}{@{}l}
group\ ::\ Eq\ a\ =>\ {\char 91}a{\char 93}\ ->\ {\char 91}{\char 91}a{\char 93}{\char 93}
\end{tabular}]\haddockbegindoc
The \haddockid{group} function takes a list and returns a list of lists such
 that the concatenation of the result is equal to the argument.  Moreover,
 each sublist in the result contains only equal elements.  For example,
\par
\begin{quote}
{\haddockverb\begin{verbatim}
 group "Mississippi" = ["M","i","ss","i","ss","i","pp","i"]
\end{verbatim}}
\end{quote}
It is a special case of \haddockid{groupBy}, which allows the programmer to supply
 their own equality test.
\par

\end{haddockdesc}
\begin{haddockdesc}
\item[\begin{tabular}{@{}l}
inits\ ::\ {\char 91}a{\char 93}\ ->\ {\char 91}{\char 91}a{\char 93}{\char 93}
\end{tabular}]\haddockbegindoc
The \haddockid{inits} function returns all initial segments of the argument,
 shortest first.  For example,
\par
\begin{quote}
{\haddockverb\begin{verbatim}
 inits "abc" == ["","a","ab","abc"]
\end{verbatim}}
\end{quote}

\end{haddockdesc}
\begin{haddockdesc}
\item[\begin{tabular}{@{}l}
tails\ ::\ {\char 91}a{\char 93}\ ->\ {\char 91}{\char 91}a{\char 93}{\char 93}
\end{tabular}]\haddockbegindoc
The \haddockid{tails} function returns all final segments of the argument,
 longest first.  For example,
\par
\begin{quote}
{\haddockverb\begin{verbatim}
 tails "abc" == ["abc", "bc", "c",""]
\end{verbatim}}
\end{quote}

\end{haddockdesc}
\subsection{Predicates
}
\begin{haddockdesc}
\item[\begin{tabular}{@{}l}
isPrefixOf\ ::\ Eq\ a\ =>\ {\char 91}a{\char 93}\ ->\ {\char 91}a{\char 93}\ ->\ Bool
\end{tabular}]\haddockbegindoc
The \haddockid{isPrefixOf} function takes two lists and returns \haddockid{True}
 iff the first list is a prefix of the second.
\par

\end{haddockdesc}
\begin{haddockdesc}
\item[\begin{tabular}{@{}l}
isSuffixOf\ ::\ Eq\ a\ =>\ {\char 91}a{\char 93}\ ->\ {\char 91}a{\char 93}\ ->\ Bool
\end{tabular}]\haddockbegindoc
The \haddockid{isSuffixOf} function takes two lists and returns \haddockid{True}
 iff the first list is a suffix of the second.
 Both lists must be finite.
\par

\end{haddockdesc}
\begin{haddockdesc}
\item[\begin{tabular}{@{}l}
isInfixOf\ ::\ Eq\ a\ =>\ {\char 91}a{\char 93}\ ->\ {\char 91}a{\char 93}\ ->\ Bool
\end{tabular}]\haddockbegindoc
The \haddockid{isInfixOf} function takes two lists and returns \haddockid{True}
 iff the first list is contained, wholly and intact,
 anywhere within the second.
\par
Example:
\par
\begin{quote}
{\haddockverb\begin{verbatim}
isInfixOf "Haskell" "I really like Haskell." == True
isInfixOf "Ial" "I really like Haskell." == False
\end{verbatim}}
\end{quote}

\end{haddockdesc}
\section{Searching lists
}
\subsection{Searching by equality
}
\begin{haddockdesc}
\item[\begin{tabular}{@{}l}
elem\ ::\ Eq\ a\ =>\ a\ ->\ {\char 91}a{\char 93}\ ->\ Bool
\end{tabular}]\haddockbegindoc
\haddockid{elem} is the list membership predicate, usually written in infix form,
 e.g., \haddocktt{x\ `elem`\ xs}.
\par

\end{haddockdesc}
\begin{haddockdesc}
\item[\begin{tabular}{@{}l}
notElem\ ::\ Eq\ a\ =>\ a\ ->\ {\char 91}a{\char 93}\ ->\ Bool
\end{tabular}]\haddockbegindoc
\haddockid{notElem} is the negation of \haddockid{elem}.
\par

\end{haddockdesc}
\begin{haddockdesc}
\item[\begin{tabular}{@{}l}
lookup\ ::\ Eq\ a\ =>\ a\ ->\ {\char 91}(a,\ b){\char 93}\ ->\ Maybe\ b
\end{tabular}]\haddockbegindoc
\haddockid{lookup} \haddocktt{key\ assocs} looks up a key in an association list.
\par

\end{haddockdesc}
\subsection{Searching with a predicate
}
\begin{haddockdesc}
\item[\begin{tabular}{@{}l}
find\ ::\ (a\ ->\ Bool)\ ->\ {\char 91}a{\char 93}\ ->\ Maybe\ a
\end{tabular}]\haddockbegindoc
The \haddockid{find} function takes a predicate and a list and returns the
 first element in the list matching the predicate, or \haddockid{Nothing} if
 there is no such element.
\par

\end{haddockdesc}
\begin{haddockdesc}
\item[\begin{tabular}{@{}l}
filter\ ::\ (a\ ->\ Bool)\ ->\ {\char 91}a{\char 93}\ ->\ {\char 91}a{\char 93}
\end{tabular}]\haddockbegindoc
\haddockid{filter}, applied to a predicate and a list, returns the list of
 those elements that satisfy the predicate; i.e.,
\par
\begin{quote}
{\haddockverb\begin{verbatim}
 filter p xs = [ x | x <- xs, p x]
\end{verbatim}}
\end{quote}

\end{haddockdesc}
\begin{haddockdesc}
\item[\begin{tabular}{@{}l}
partition\ ::\ (a\ ->\ Bool)\ ->\ {\char 91}a{\char 93}\ ->\ ({\char 91}a{\char 93},\ {\char 91}a{\char 93})
\end{tabular}]\haddockbegindoc
The \haddockid{partition} function takes a predicate a list and returns
 the pair of lists of elements which do and do not satisfy the
 predicate, respectively; i.e.,
\par
\begin{quote}
{\haddockverb\begin{verbatim}
 partition p xs == (filter p xs, filter (not . p) xs)
\end{verbatim}}
\end{quote}

\end{haddockdesc}
\section{Indexing lists
}
These functions treat a list \haddocktt{xs} as a indexed collection,
 with indices ranging from 0 to \haddocktt{length\ xs\ -\ 1}.
\par

\begin{haddockdesc}
\item[\begin{tabular}{@{}l}
(!!)\ ::\ {\char 91}a{\char 93}\ ->\ Int\ ->\ a
\end{tabular}]\haddockbegindoc
List index (subscript) operator, starting from 0.
 It is an instance of the more general \haddocktt{Data.List.genericIndex},
 which takes an index of any integral type.
\par

\end{haddockdesc}
\begin{haddockdesc}
\item[\begin{tabular}{@{}l}
elemIndex\ ::\ Eq\ a\ =>\ a\ ->\ {\char 91}a{\char 93}\ ->\ Maybe\ Int
\end{tabular}]\haddockbegindoc
The \haddockid{elemIndex} function returns the index of the first element
 in the given list which is equal (by \haddockid{==}) to the query element,
 or \haddockid{Nothing} if there is no such element.
\par

\end{haddockdesc}
\begin{haddockdesc}
\item[\begin{tabular}{@{}l}
elemIndices\ ::\ Eq\ a\ =>\ a\ ->\ {\char 91}a{\char 93}\ ->\ {\char 91}Int{\char 93}
\end{tabular}]\haddockbegindoc
The \haddockid{elemIndices} function extends \haddockid{elemIndex}, by returning the
 indices of all elements equal to the query element, in ascending order.
\par

\end{haddockdesc}
\begin{haddockdesc}
\item[\begin{tabular}{@{}l}
findIndex\ ::\ (a\ ->\ Bool)\ ->\ {\char 91}a{\char 93}\ ->\ Maybe\ Int
\end{tabular}]\haddockbegindoc
The \haddockid{findIndex} function takes a predicate and a list and returns
 the index of the first element in the list satisfying the predicate,
 or \haddockid{Nothing} if there is no such element.
\par

\end{haddockdesc}
\begin{haddockdesc}
\item[\begin{tabular}{@{}l}
findIndices\ ::\ (a\ ->\ Bool)\ ->\ {\char 91}a{\char 93}\ ->\ {\char 91}Int{\char 93}
\end{tabular}]\haddockbegindoc
The \haddockid{findIndices} function extends \haddockid{findIndex}, by returning the
 indices of all elements satisfying the predicate, in ascending order.
\par

\end{haddockdesc}
\section{Zipping and unzipping lists
}
\begin{haddockdesc}
\item[\begin{tabular}{@{}l}
zip\ ::\ {\char 91}a{\char 93}\ ->\ {\char 91}b{\char 93}\ ->\ {\char 91}(a,\ b){\char 93}
\end{tabular}]\haddockbegindoc
\haddockid{zip} takes two lists and returns a list of corresponding pairs.
 If one input list is short, excess elements of the longer list are
 discarded.
\par

\end{haddockdesc}
\begin{haddockdesc}
\item[\begin{tabular}{@{}l}
zip3\ ::\ {\char 91}a{\char 93}\ ->\ {\char 91}b{\char 93}\ ->\ {\char 91}c{\char 93}\ ->\ {\char 91}(a,\ b,\ c){\char 93}
\end{tabular}]\haddockbegindoc
\haddockid{zip3} takes three lists and returns a list of triples, analogous to
 \haddockid{zip}.
\par

\end{haddockdesc}
\begin{haddockdesc}
\item[\begin{tabular}{@{}l}
zip4\ ::\ {\char 91}a{\char 93}\ ->\ {\char 91}b{\char 93}\ ->\ {\char 91}c{\char 93}\ ->\ {\char 91}d{\char 93}\ ->\ {\char 91}(a,\ b,\ c,\ d){\char 93}
\end{tabular}]\haddockbegindoc
The \haddockid{zip4} function takes four lists and returns a list of
 quadruples, analogous to \haddockid{zip}.
\par

\end{haddockdesc}
\begin{haddockdesc}
\item[\begin{tabular}{@{}l}
zip5\ ::\ {\char 91}a{\char 93}\ ->\ {\char 91}b{\char 93}\ ->\ {\char 91}c{\char 93}\ ->\ {\char 91}d{\char 93}\ ->\ {\char 91}e{\char 93}\ ->\ {\char 91}(a,\ b,\ c,\ d,\ e){\char 93}
\end{tabular}]\haddockbegindoc
The \haddockid{zip5} function takes five lists and returns a list of
 five-tuples, analogous to \haddockid{zip}.
\par

\end{haddockdesc}
\begin{haddockdesc}
\item[\begin{tabular}{@{}l}
zip6\ ::\ {\char 91}a{\char 93}\\\ \ \ \ \ \ \ \ ->\ {\char 91}b{\char 93}\ ->\ {\char 91}c{\char 93}\ ->\ {\char 91}d{\char 93}\ ->\ {\char 91}e{\char 93}\ ->\ {\char 91}f{\char 93}\ ->\ {\char 91}(a,\ b,\ c,\ d,\ e,\ f){\char 93}
\end{tabular}]\haddockbegindoc
The \haddockid{zip6} function takes six lists and returns a list of six-tuples,
 analogous to \haddockid{zip}.
\par

\end{haddockdesc}
\begin{haddockdesc}
\item[\begin{tabular}{@{}l}
zip7\ ::\ {\char 91}a{\char 93}\\\ \ \ \ \ \ \ \ ->\ {\char 91}b{\char 93}\\\ \ \ \ \ \ \ \ \ \ \ ->\ {\char 91}c{\char 93}\ ->\ {\char 91}d{\char 93}\ ->\ {\char 91}e{\char 93}\ ->\ {\char 91}f{\char 93}\ ->\ {\char 91}g{\char 93}\ ->\ {\char 91}(a,\ b,\ c,\ d,\ e,\ f,\ g){\char 93}
\end{tabular}]\haddockbegindoc
The \haddockid{zip7} function takes seven lists and returns a list of
 seven-tuples, analogous to \haddockid{zip}.
\par

\end{haddockdesc}
\begin{haddockdesc}
\item[\begin{tabular}{@{}l}
zipWith\ ::\ (a\ ->\ b\ ->\ c)\ ->\ {\char 91}a{\char 93}\ ->\ {\char 91}b{\char 93}\ ->\ {\char 91}c{\char 93}
\end{tabular}]\haddockbegindoc
\haddockid{zipWith} generalises \haddockid{zip} by zipping with the function given
 as the first argument, instead of a tupling function.
 For example, \haddocktt{zipWith\ (+)} is applied to two lists to produce the
 list of corresponding sums.
\par

\end{haddockdesc}
\begin{haddockdesc}
\item[\begin{tabular}{@{}l}
zipWith3\ ::\ (a\ ->\ b\ ->\ c\ ->\ d)\ ->\ {\char 91}a{\char 93}\ ->\ {\char 91}b{\char 93}\ ->\ {\char 91}c{\char 93}\ ->\ {\char 91}d{\char 93}
\end{tabular}]\haddockbegindoc
The \haddockid{zipWith3} function takes a function which combines three
 elements, as well as three lists and returns a list of their point-wise
 combination, analogous to \haddockid{zipWith}.
\par

\end{haddockdesc}
\begin{haddockdesc}
\item[\begin{tabular}{@{}l}
zipWith4\ ::\ (a\ ->\ b\ ->\ c\ ->\ d\ ->\ e)\\\ \ \ \ \ \ \ \ \ \ \ \ ->\ {\char 91}a{\char 93}\ ->\ {\char 91}b{\char 93}\ ->\ {\char 91}c{\char 93}\ ->\ {\char 91}d{\char 93}\ ->\ {\char 91}e{\char 93}
\end{tabular}]\haddockbegindoc
The \haddockid{zipWith4} function takes a function which combines four
 elements, as well as four lists and returns a list of their point-wise
 combination, analogous to \haddockid{zipWith}.
\par

\end{haddockdesc}
\begin{haddockdesc}
\item[\begin{tabular}{@{}l}
zipWith5\ ::\ (a\ ->\ b\ ->\ c\ ->\ d\ ->\ e\ ->\ f)\\\ \ \ \ \ \ \ \ \ \ \ \ ->\ {\char 91}a{\char 93}\ ->\ {\char 91}b{\char 93}\ ->\ {\char 91}c{\char 93}\ ->\ {\char 91}d{\char 93}\ ->\ {\char 91}e{\char 93}\ ->\ {\char 91}f{\char 93}
\end{tabular}]\haddockbegindoc
The \haddockid{zipWith5} function takes a function which combines five
 elements, as well as five lists and returns a list of their point-wise
 combination, analogous to \haddockid{zipWith}.
\par

\end{haddockdesc}
\begin{haddockdesc}
\item[\begin{tabular}{@{}l}
zipWith6\ ::\ (a\ ->\ b\ ->\ c\ ->\ d\ ->\ e\ ->\ f\ ->\ g)\\\ \ \ \ \ \ \ \ \ \ \ \ ->\ {\char 91}a{\char 93}\ ->\ {\char 91}b{\char 93}\ ->\ {\char 91}c{\char 93}\ ->\ {\char 91}d{\char 93}\ ->\ {\char 91}e{\char 93}\ ->\ {\char 91}f{\char 93}\ ->\ {\char 91}g{\char 93}
\end{tabular}]\haddockbegindoc
The \haddockid{zipWith6} function takes a function which combines six
 elements, as well as six lists and returns a list of their point-wise
 combination, analogous to \haddockid{zipWith}.
\par

\end{haddockdesc}
\begin{haddockdesc}
\item[\begin{tabular}{@{}l}
zipWith7\ ::\ (a\ ->\ b\ ->\ c\ ->\ d\ ->\ e\ ->\ f\ ->\ g\ ->\ h)\\\ \ \ \ \ \ \ \ \ \ \ \ ->\ {\char 91}a{\char 93}\ ->\ {\char 91}b{\char 93}\ ->\ {\char 91}c{\char 93}\ ->\ {\char 91}d{\char 93}\ ->\ {\char 91}e{\char 93}\ ->\ {\char 91}f{\char 93}\ ->\ {\char 91}g{\char 93}\ ->\ {\char 91}h{\char 93}
\end{tabular}]\haddockbegindoc
The \haddockid{zipWith7} function takes a function which combines seven
 elements, as well as seven lists and returns a list of their point-wise
 combination, analogous to \haddockid{zipWith}.
\par

\end{haddockdesc}
\begin{haddockdesc}
\item[\begin{tabular}{@{}l}
unzip\ ::\ {\char 91}(a,\ b){\char 93}\ ->\ ({\char 91}a{\char 93},\ {\char 91}b{\char 93})
\end{tabular}]\haddockbegindoc
\haddockid{unzip} transforms a list of pairs into a list of first components
 and a list of second components.
\par

\end{haddockdesc}
\begin{haddockdesc}
\item[\begin{tabular}{@{}l}
unzip3\ ::\ {\char 91}(a,\ b,\ c){\char 93}\ ->\ ({\char 91}a{\char 93},\ {\char 91}b{\char 93},\ {\char 91}c{\char 93})
\end{tabular}]\haddockbegindoc
The \haddockid{unzip3} function takes a list of triples and returns three
 lists, analogous to \haddockid{unzip}.
\par

\end{haddockdesc}
\begin{haddockdesc}
\item[\begin{tabular}{@{}l}
unzip4\ ::\ {\char 91}(a,\ b,\ c,\ d){\char 93}\ ->\ ({\char 91}a{\char 93},\ {\char 91}b{\char 93},\ {\char 91}c{\char 93},\ {\char 91}d{\char 93})
\end{tabular}]\haddockbegindoc
The \haddockid{unzip4} function takes a list of quadruples and returns four
 lists, analogous to \haddockid{unzip}.
\par

\end{haddockdesc}
\begin{haddockdesc}
\item[\begin{tabular}{@{}l}
unzip5\ ::\ {\char 91}(a,\ b,\ c,\ d,\ e){\char 93}\ ->\ ({\char 91}a{\char 93},\ {\char 91}b{\char 93},\ {\char 91}c{\char 93},\ {\char 91}d{\char 93},\ {\char 91}e{\char 93})
\end{tabular}]\haddockbegindoc
The \haddockid{unzip5} function takes a list of five-tuples and returns five
 lists, analogous to \haddockid{unzip}.
\par

\end{haddockdesc}
\begin{haddockdesc}
\item[\begin{tabular}{@{}l}
unzip6\ ::\ {\char 91}(a,\ b,\ c,\ d,\ e,\ f){\char 93}\ ->\ ({\char 91}a{\char 93},\ {\char 91}b{\char 93},\ {\char 91}c{\char 93},\ {\char 91}d{\char 93},\ {\char 91}e{\char 93},\ {\char 91}f{\char 93})
\end{tabular}]\haddockbegindoc
The \haddockid{unzip6} function takes a list of six-tuples and returns six
 lists, analogous to \haddockid{unzip}.
\par

\end{haddockdesc}
\begin{haddockdesc}
\item[\begin{tabular}{@{}l}
unzip7\ ::\ {\char 91}(a,\ b,\ c,\ d,\ e,\ f,\ g){\char 93}\\\ \ \ \ \ \ \ \ \ \ ->\ ({\char 91}a{\char 93},\ {\char 91}b{\char 93},\ {\char 91}c{\char 93},\ {\char 91}d{\char 93},\ {\char 91}e{\char 93},\ {\char 91}f{\char 93},\ {\char 91}g{\char 93})
\end{tabular}]\haddockbegindoc
The \haddockid{unzip7} function takes a list of seven-tuples and returns
 seven lists, analogous to \haddockid{unzip}.
\par

\end{haddockdesc}
\section{Special lists
}
\subsection{Functions on strings
}
\begin{haddockdesc}
\item[\begin{tabular}{@{}l}
lines\ ::\ String\ ->\ {\char 91}String{\char 93}
\end{tabular}]\haddockbegindoc
\haddockid{lines} breaks a string up into a list of strings at newline
 characters.  The resulting strings do not contain newlines.
\par

\end{haddockdesc}
\begin{haddockdesc}
\item[\begin{tabular}{@{}l}
words\ ::\ String\ ->\ {\char 91}String{\char 93}
\end{tabular}]\haddockbegindoc
\haddockid{words} breaks a string up into a list of words, which were delimited
 by white space.
\par

\end{haddockdesc}
\begin{haddockdesc}
\item[\begin{tabular}{@{}l}
unlines\ ::\ {\char 91}String{\char 93}\ ->\ String
\end{tabular}]\haddockbegindoc
\haddockid{unlines} is an inverse operation to \haddockid{lines}.
 It joins lines, after appending a terminating newline to each.
\par

\end{haddockdesc}
\begin{haddockdesc}
\item[\begin{tabular}{@{}l}
unwords\ ::\ {\char 91}String{\char 93}\ ->\ String
\end{tabular}]\haddockbegindoc
\haddockid{unwords} is an inverse operation to \haddockid{words}.
 It joins words with separating spaces.
\par

\end{haddockdesc}
\subsection{"Set" operations
}
\begin{haddockdesc}
\item[\begin{tabular}{@{}l}
nub\ ::\ Eq\ a\ =>\ {\char 91}a{\char 93}\ ->\ {\char 91}a{\char 93}
\end{tabular}]\haddockbegindoc
\emph{O(n{\char '136}2)}. The \haddockid{nub} function removes duplicate elements from a list.
 In particular, it keeps only the first occurrence of each element.
 (The name \haddockid{nub} means `essence'.)
 It is a special case of \haddockid{nubBy}, which allows the programmer to supply
 their own equality test.
\par

\end{haddockdesc}
\begin{haddockdesc}
\item[\begin{tabular}{@{}l}
delete\ ::\ Eq\ a\ =>\ a\ ->\ {\char 91}a{\char 93}\ ->\ {\char 91}a{\char 93}
\end{tabular}]\haddockbegindoc
\haddockid{delete} \haddocktt{x} removes the first occurrence of \haddocktt{x} from its list argument.
 For example,
\par
\begin{quote}
{\haddockverb\begin{verbatim}
 delete 'a' "banana" == "bnana"
\end{verbatim}}
\end{quote}
It is a special case of \haddockid{deleteBy}, which allows the programmer to
 supply their own equality test.
\par

\end{haddockdesc}
\begin{haddockdesc}
\item[\begin{tabular}{@{}l}
({\char '134}{\char '134})\ ::\ Eq\ a\ =>\ {\char 91}a{\char 93}\ ->\ {\char 91}a{\char 93}\ ->\ {\char 91}a{\char 93}
\end{tabular}]\haddockbegindoc
The \haddockid{{\char '134}{\char '134}} function is list difference ((non-associative).
 In the result of \haddocktt{xs} \haddockid{{\char '134}{\char '134}} \haddocktt{ys}, the first occurrence of each element of
 \haddocktt{ys} in turn (if any) has been removed from \haddocktt{xs}.  Thus
\par
\begin{quote}
{\haddockverb\begin{verbatim}
 (xs ++ ys) \\ xs == ys.
\end{verbatim}}
\end{quote}
It is a special case of \haddockid{deleteFirstsBy}, which allows the programmer
 to supply their own equality test.
\par

\end{haddockdesc}
\begin{haddockdesc}
\item[\begin{tabular}{@{}l}
union\ ::\ Eq\ a\ =>\ {\char 91}a{\char 93}\ ->\ {\char 91}a{\char 93}\ ->\ {\char 91}a{\char 93}
\end{tabular}]\haddockbegindoc
The \haddockid{union} function returns the list union of the two lists.
 For example,
\par
\begin{quote}
{\haddockverb\begin{verbatim}
 "dog" `union` "cow" == "dogcw"
\end{verbatim}}
\end{quote}
Duplicates, and elements of the first list, are removed from the
 the second list, but if the first list contains duplicates, so will
 the result.
 It is a special case of \haddockid{unionBy}, which allows the programmer to supply
 their own equality test.
\par

\end{haddockdesc}
\begin{haddockdesc}
\item[\begin{tabular}{@{}l}
intersect\ ::\ Eq\ a\ =>\ {\char 91}a{\char 93}\ ->\ {\char 91}a{\char 93}\ ->\ {\char 91}a{\char 93}
\end{tabular}]\haddockbegindoc
The \haddockid{intersect} function takes the list intersection of two lists.
 For example,
\par
\begin{quote}
{\haddockverb\begin{verbatim}
 [1,2,3,4] `intersect` [2,4,6,8] == [2,4]
\end{verbatim}}
\end{quote}
If the first list contains duplicates, so will the result.
\par
\begin{quote}
{\haddockverb\begin{verbatim}
 [1,2,2,3,4] `intersect` [6,4,4,2] == [2,2,4]
\end{verbatim}}
\end{quote}
It is a special case of \haddockid{intersectBy}, which allows the programmer to
 supply their own equality test.
\par

\end{haddockdesc}
\subsection{Ordered lists
}
\begin{haddockdesc}
\item[\begin{tabular}{@{}l}
sort\ ::\ Ord\ a\ =>\ {\char 91}a{\char 93}\ ->\ {\char 91}a{\char 93}
\end{tabular}]\haddockbegindoc
The \haddockid{sort} function implements a stable sorting algorithm.
 It is a special case of \haddockid{sortBy}, which allows the programmer to supply
 their own comparison function.
\par

\end{haddockdesc}
\begin{haddockdesc}
\item[\begin{tabular}{@{}l}
insert\ ::\ Ord\ a\ =>\ a\ ->\ {\char 91}a{\char 93}\ ->\ {\char 91}a{\char 93}
\end{tabular}]\haddockbegindoc
The \haddockid{insert} function takes an element and a list and inserts the
 element into the list at the last position where it is still less
 than or equal to the next element.  In particular, if the list
 is sorted before the call, the result will also be sorted.
 It is a special case of \haddockid{insertBy}, which allows the programmer to
 supply their own comparison function.
\par

\end{haddockdesc}
\section{Generalized functions
}
\subsection{The "\haddocktt{By}" operations
}
By convention, overloaded functions have a non-overloaded
 counterpart whose name is suffixed with `\haddocktt{By}'.
\par

\subsubsection{User-supplied equality (replacing an \haddocktt{Eq} context)
}
The predicate is assumed to define an equivalence.
\par

\begin{haddockdesc}
\item[\begin{tabular}{@{}l}
nubBy\ ::\ (a\ ->\ a\ ->\ Bool)\ ->\ {\char 91}a{\char 93}\ ->\ {\char 91}a{\char 93}
\end{tabular}]\haddockbegindoc
The \haddockid{nubBy} function behaves just like \haddockid{nub}, except it uses a
 user-supplied equality predicate instead of the overloaded \haddockid{==}
 function.
\par

\end{haddockdesc}
\begin{haddockdesc}
\item[\begin{tabular}{@{}l}
deleteBy\ ::\ (a\ ->\ a\ ->\ Bool)\ ->\ a\ ->\ {\char 91}a{\char 93}\ ->\ {\char 91}a{\char 93}
\end{tabular}]\haddockbegindoc
The \haddockid{deleteBy} function behaves like \haddockid{delete}, but takes a
 user-supplied equality predicate.
\par

\end{haddockdesc}
\begin{haddockdesc}
\item[\begin{tabular}{@{}l}
deleteFirstsBy\ ::\ (a\ ->\ a\ ->\ Bool)\ ->\ {\char 91}a{\char 93}\ ->\ {\char 91}a{\char 93}\ ->\ {\char 91}a{\char 93}
\end{tabular}]\haddockbegindoc
The \haddockid{deleteFirstsBy} function takes a predicate and two lists and
 returns the first list with the first occurrence of each element of
 the second list removed.
\par

\end{haddockdesc}
\begin{haddockdesc}
\item[\begin{tabular}{@{}l}
unionBy\ ::\ (a\ ->\ a\ ->\ Bool)\ ->\ {\char 91}a{\char 93}\ ->\ {\char 91}a{\char 93}\ ->\ {\char 91}a{\char 93}
\end{tabular}]\haddockbegindoc
The \haddockid{unionBy} function is the non-overloaded version of \haddockid{union}.
\par

\end{haddockdesc}
\begin{haddockdesc}
\item[\begin{tabular}{@{}l}
intersectBy\ ::\ (a\ ->\ a\ ->\ Bool)\ ->\ {\char 91}a{\char 93}\ ->\ {\char 91}a{\char 93}\ ->\ {\char 91}a{\char 93}
\end{tabular}]\haddockbegindoc
The \haddockid{intersectBy} function is the non-overloaded version of \haddockid{intersect}.
\par

\end{haddockdesc}
\begin{haddockdesc}
\item[\begin{tabular}{@{}l}
groupBy\ ::\ (a\ ->\ a\ ->\ Bool)\ ->\ {\char 91}a{\char 93}\ ->\ {\char 91}{\char 91}a{\char 93}{\char 93}
\end{tabular}]\haddockbegindoc
The \haddockid{groupBy} function is the non-overloaded version of \haddockid{group}.
\par

\end{haddockdesc}
\subsubsection{User-supplied comparison (replacing an \haddocktt{Ord} context)
}
The function is assumed to define a total ordering.
\par

\begin{haddockdesc}
\item[\begin{tabular}{@{}l}
sortBy\ ::\ (a\ ->\ a\ ->\ Ordering)\ ->\ {\char 91}a{\char 93}\ ->\ {\char 91}a{\char 93}
\end{tabular}]\haddockbegindoc
The \haddockid{sortBy} function is the non-overloaded version of \haddockid{sort}.
\par

\end{haddockdesc}
\begin{haddockdesc}
\item[\begin{tabular}{@{}l}
insertBy\ ::\ (a\ ->\ a\ ->\ Ordering)\ ->\ a\ ->\ {\char 91}a{\char 93}\ ->\ {\char 91}a{\char 93}
\end{tabular}]\haddockbegindoc
The non-overloaded version of \haddockid{insert}.
\par

\end{haddockdesc}
\begin{haddockdesc}
\item[\begin{tabular}{@{}l}
maximumBy\ ::\ (a\ ->\ a\ ->\ Ordering)\ ->\ {\char 91}a{\char 93}\ ->\ a
\end{tabular}]\haddockbegindoc
The \haddockid{maximumBy} function takes a comparison function and a list
 and returns the greatest element of the list by the comparison function.
 The list must be finite and non-empty.
\par

\end{haddockdesc}
\begin{haddockdesc}
\item[\begin{tabular}{@{}l}
minimumBy\ ::\ (a\ ->\ a\ ->\ Ordering)\ ->\ {\char 91}a{\char 93}\ ->\ a
\end{tabular}]\haddockbegindoc
The \haddockid{minimumBy} function takes a comparison function and a list
 and returns the least element of the list by the comparison function.
 The list must be finite and non-empty.
\par

\end{haddockdesc}
\subsection{The "\haddocktt{generic}" operations
}
The prefix `\haddocktt{generic}' indicates an overloaded function that
 is a generalized version of a \haddocktt{Prelude} function.
\par

\begin{haddockdesc}
\item[\begin{tabular}{@{}l}
genericLength\ ::\ Num\ i\ =>\ {\char 91}b{\char 93}\ ->\ i
\end{tabular}]\haddockbegindoc
The \haddockid{genericLength} function is an overloaded version of \haddockid{length}.  In
 particular, instead of returning an \haddockid{Int}, it returns any type which is
 an instance of \haddockid{Num}.  It is, however, less efficient than \haddockid{length}.
\par

\end{haddockdesc}
\begin{haddockdesc}
\item[\begin{tabular}{@{}l}
genericTake\ ::\ Integral\ i\ =>\ i\ ->\ {\char 91}a{\char 93}\ ->\ {\char 91}a{\char 93}
\end{tabular}]\haddockbegindoc
The \haddockid{genericTake} function is an overloaded version of \haddockid{take}, which
 accepts any \haddockid{Integral} value as the number of elements to take.
\par

\end{haddockdesc}
\begin{haddockdesc}
\item[\begin{tabular}{@{}l}
genericDrop\ ::\ Integral\ i\ =>\ i\ ->\ {\char 91}a{\char 93}\ ->\ {\char 91}a{\char 93}
\end{tabular}]\haddockbegindoc
The \haddockid{genericDrop} function is an overloaded version of \haddockid{drop}, which
 accepts any \haddockid{Integral} value as the number of elements to drop.
\par

\end{haddockdesc}
\begin{haddockdesc}
\item[\begin{tabular}{@{}l}
genericSplitAt\ ::\ Integral\ i\ =>\ i\ ->\ {\char 91}b{\char 93}\ ->\ ({\char 91}b{\char 93},\ {\char 91}b{\char 93})
\end{tabular}]\haddockbegindoc
The \haddockid{genericSplitAt} function is an overloaded version of \haddockid{splitAt}, which
 accepts any \haddockid{Integral} value as the position at which to split.
\par

\end{haddockdesc}
\begin{haddockdesc}
\item[\begin{tabular}{@{}l}
genericIndex\ ::\ Integral\ a\ =>\ {\char 91}b{\char 93}\ ->\ a\ ->\ b
\end{tabular}]\haddockbegindoc
The \haddockid{genericIndex} function is an overloaded version of \haddockid{!!}, which
 accepts any \haddockid{Integral} value as the index.
\par

\end{haddockdesc}
\begin{haddockdesc}
\item[\begin{tabular}{@{}l}
genericReplicate\ ::\ Integral\ i\ =>\ i\ ->\ a\ ->\ {\char 91}a{\char 93}
\end{tabular}]\haddockbegindoc
The \haddockid{genericReplicate} function is an overloaded version of \haddockid{replicate},
 which accepts any \haddockid{Integral} value as the number of repetitions to make.
\par

\end{haddockdesc}